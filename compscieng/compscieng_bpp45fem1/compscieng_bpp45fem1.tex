\documentclass[12pt,fleqn]{article}\usepackage{../../common}
\begin{document}
Sonlu Öğeler Metotu (Finite Elements Method -FEM-) - 1

Galerkin, Ağırlıklı Artıklar Metotu (Weighted Residual Method)

WRS diferansiyel denklemleri yaklaşık (approximate) olarak çözebilen bir
metottur, bunu tüm yaklaşık fonksiyonun tanım kümesi üzerinden ağırlıklı hata
artıklarını sıfıra eşitleyerek yapar. Bununla ne demek istediğimizi birazdan
yakından göreceğiz [1, Ders 3].

Örnek olarak [2]'de öğrendiğimiz Euler-Bernoulli kirişlerini tanımlayan denklemi
hatırlarsak,

$$
E I \frac{\ud^4 y}{\ud X_1^4} = q
$$

Biraz düzenleme sonrası

$$
E I \frac{\ud^4 y}{\ud X_1^4} - q = 0
$$

elde ederim. Amacım öyle bir yaklaşık $y$, ya da $y_{approx}$ diyelim, bulmak ki
üstteki denklemi çözebileyim. Bunu $y$ yerine onu yaklaşık temsil edebilen bir
diğer fonksiyonu geçirerek yapabilirim. Bir polinom bu işi görebilir; Pek çok
diğer yöntemin kullandığı tipik bir polinom vardır,

$$
y_{approx} = a_0 + a_1 X_1 + a_2 X_2^2 
$$

diye gider, aslında daha genel olarak olan her terimde ``bir katsayı çarpı
$X_1$'in bir tür fonksiyonu'' gibi bir toplam kullanmak daha iyi olabilir,
bu formda,

$$
y_{approx} = a_0 \phi_0(X_1) + c_1 \phi_1(X_1) + c_2 \phi_2(X_2) 
$$












[devam edecek]

Kaynaklar

[1] Petitt, {\em Intro to the Finite Element Method}, University of Alberta,
    \url{https://www.youtube.com/watch?v=2iUnfPRk6Ro&list=PLLSzlda_AXa3yQEJAb5JcmsVDy9i9K_fi}

[2] Bayramlı, {\em Fizik, Materyel Mekanigi - 2}
    
\end{document}





