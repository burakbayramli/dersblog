


Zayıf Form

Daha önce diferansiyel formun her durumda çözülemediğinden bahsettik. Fakat
(4)'teki entegral form da çalışılması zor bir halde. Daha rahat çalışılabilen
bir entegral forma erisemez miyiz? Bu forma erişmek için diferansiyel
denklemlerde bazen kullanılan bir tekniği kullanacağız, bu teknikle çözümü
tanımlamak için daha az pürüzsüzlük istediğimiz durumlarda diferansiyel denklemi
tekrar yazıyoruz. Tekrar yazmak için denklemi alıp onu bir pürüzsüz ``test
fonksiyonu'' ile çarpıyoruz, bu çarpımı birkaç kere bir tanım kümesi üzerinden
entegre ediyoruz, ve Parçalı Entegral tekniğini kullanarak türevi $u$'dan alıp
pürüzsüz olan test fonksiyonuna aktarıyoruz. Böylece $u$ üzerinde daha az
türev olan bir formül elde ediyoruz, ve $u$'nun pürüzsüz olma şartı gevşemiş
oluyor [3, sf. 27].

Test fonksiyonu hem zamanda hem uzayda pürüzsüz bir fonksiyon $\phi$ olsun,
ayrıca derlitoplu desteğe sahip (compactly supported), yani belli bir kısıtlı
alan dışında sıfır, bu örnekte sonsuzluğun kendisi alanın dışında kabul edilir
ve o noktada $\phi$ sıfır olur [4]. Bu ileride bazı iptalleri yaparken işimize
yarayacak.

(1) formülünden başlayalım,

$$
u_t + f(u)_x = 0
$$

Bu fonksiyonu $\phi$ ile çarpıp uzay ve zaman üzerinden entegre edersek,

$$
\int _{0}^{\infty} \int _{-\infty}^{\infty} u_t \phi + f(u)_x \phi \ud x \ud t = 0
\mlabel{5}
$$

elde ederiz. Bu denklemdeki ilk terime zaman üzerinden, ikinci terime de uzay
üzerinden Parçalı Entegral tekniğini uygulayacağız. PE tekniğini hatırlarsak,

$$
\int y \ud z = y z - \int z \ud y
$$

eşitliği idi, ya da belirli entegraller (definite integrals) için,

$$
\int_{a}^{b} y \ud z = [y z]_{a}^{b} - \int_{a}^{b} z \ud y
\mlabel{6}
$$

Kısım olarak $y = \phi \ud x$, $\ud z = u_t$ atayalım, şimdi zaman üzerinden PE
alıyoruz, dikkat bu entegral dışarıdan içeri doğru yani zaman için entegral
(5)'in tüm iç kısmı

$$
\int _{0}^{\infty}
\underbrace{\int _{-\infty}^{\infty} u_t \phi}_{dx dt\textrm{ ile beraber iç kısım}} + f(u)_x \phi \ud x
\ud t = 0
$$

üzerinde alınmış olacak. Dedigimiz gibi once $y,\ud z$ tanımladık, o zaman $z=u$,
$\ud y = \phi_t \ud x \ud t$ olur. (6) ile yerine koyarsak,

$$
\int_{-\infty}^{\infty} [ u \phi ]_{0}^{\infty} -
\int _{0}^{\infty} \int_{-\infty}^{\infty} u \phi_t \ud x \ud t
\mlabel{7}
$$

Test fonksiyonu tanımından $\phi(\pm \infty, t) = \phi(x, \pm \infty) = 0$
olduğunu biliyoruz, yani

$$
[ u \phi ]_{0}^{\infty} = \cancel{u(x,\infty)\phi(x,\infty)} - u(x,0)\phi(x,0) 
$$

$$
= u(x,0)\phi(x,0) 
$$

Bu sayede (7) basitleştirilir,

$$
- \int _{0}^{\infty} \int_{-\infty}^{\infty} u \phi_t \ud x \ud t +
\int_{-\infty}^{\infty} u(x,0)\phi(x,0) 
$$

Benzer bir yaklaşımla zaman üzerinden PE alınır [4], ve

$$
\int_{0}^{\infty} \int_{-\infty}^{\infty}
f(u)_x \phi \ud x \ud t =
- \int_{0}^{\infty} \int_{-\infty}^{\infty} f(u) \phi_x \ud x \ud t
$$

Son iki formül birleştirilince,

$$
\int_{0}^{\infty} \int_{-\infty}^{\infty}
u \phi_t + f(u) \phi_x \ud x \ud t = -
\int_{-\infty}^{\infty} u(x,0) \phi(x,0) \ud x 
$$

Burgers denklemi çözümü (9)'un aynı zamanda bir zayıf çözüm olduğunun ispatı
[6, sf. 8]'de bulunabilir. Not: ispat sadece 1. seçenek $u_l > u_r$ için
geçerli, diğer seçenekler de var.



[devam edecek]



Kaynaklar

[3] Leveque, {\em Numerical Methods for Conservation Laws}

[4] Dong, {\em Numerical Methods for Conservation Laws (Part 1)}
    \url{https://jdongg.github.io/post/conservation-laws-1/}



    
