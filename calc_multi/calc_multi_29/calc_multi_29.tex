\documentclass[12pt,fleqn]{article}\usepackage{../../common}
\begin{document}
Ders 29

Uzaklaşım Teorisi bize kapalı bir alandaki vektör alanının akışını (flux)
hesaplama imkanı veriyor, 

\includegraphics[width=10em]{calc_multi_29_02.jpg}

Formülün sol tarafında yüzey üzerinden bir entegral var, UT ile onu $\vec{F}$
uzaklaşımını üzerinden üçlü entegrale çevirebiliyoruz,

$$
\int \int_S \vec{F} \ud \vec{S} =
\int \int \int_D \bdiv \vec{F} \ud V
$$

Mesela eger $\vec{F}$ bilesenleri $P,Q,R$ ise,

$$
\int \int_S < P,Q,R > \vec{n} \ud S = \int \int \int_D \left( P_x + Q_y + R_z \right) \ud V
$$

Peki Uzaklaşım Teorisinin fiziksel anlamı nedir?

Fiziksel olarak $\bdiv(\vec{F})$ kaynak üretilis oranı olarak görülebilir,
ya da birim hacimde üretilen akış.

Örnek olarak sıkıştırılamayan bir sıvıyı düşünelim, mesela şu. Suyun sabitlenmiş
bir kütlesi her zaman aynı hacmi kaplar. Gazlar sıkıştırılabilir, onlar
farklıdır.  O zaman önümüzde bir sıkıştırılamayan akış (incopressible flow)
problemi varsa, ve verili bir kütle her zaman aynı hacmi kaplıyorsa, ve elimizde
bir $\vec{F}$ hızı var ise (şu durumunda vektör alanı her noktadaki akış hızı
olarak kabul edilebilir) o zaman Uzaklaşım Teorisi şunu söylüyor, belli bir
hacimdeki $\vec{F}$ uzaklaşımlarını topla, $\int \int \int_D \bdiv \vec{F} \ud V$, 
bu sayı yüzeyin akışına eşit olacaktır, $\int \int_S \vec{F} \cdot \vec{n} \ud S$.
Akış birim zamanda $D$ bölgesini terkeden sıvı miktarı, tabi terkeden derken
giren ekşi çıkan, akışın ölçtüğü bu ikisinin farkı.

Bölge bağlamında uzaklaşımın hesapladığı $D$ bölgesi içindeki tüm kaynakların
ekşi alıcıların (sink) ürettiği / tükettiği birim hacimdeki sıvı, bunları
tüm hacim için entegre edince toplam üretilen sıvı ortaya çıkar, bu üretim
sıkıştırılamayan sıvı için tabii ki yüzeyden geçip dışarı çıkacaktır, yüzey
entegralinin de ölçtüğü budur.

Uzaklaşım Teorisi çok bariz bir şeyi belirtiyor demek ki, sınırdan geçen
her kütle bir yerden geliyor olmalı.

Uzaklaşım teorisinin ispatına gelelim. Bu ispatın daha kolay versiyonunu
yapacağım şimdi, tüm eşitlik yerine

$$
\int \int_S < 0,0,R > \cdot \hat{n} \ud S =
\int \int \int_D R_z \ud V
\mlabel{1}
$$

eşitliğinin ispatını yapacağım. Buradan hareketle daha genel eşitliği
ispatlamak kolay, aynı ispatı sadece $x$, sadece $y$ bileşeni olan
vektör alanları için tekrarlarım, ve tüm bunları toplayınca ana eşitliği
elde etmiş olurum.

İkinci bir basitleştirme yapalım, çünkü ispatı hala herhangi bir yüzey için
yapabileceğimden emin değilim. Dikey, basit bir yüzey kullanacağım, öyle ki
bu yüzey üzerinden entegralde $z$ değişkenin sınırlarını kolay halledebileyim.

\includegraphics[width=15em]{calc_multi_29_01.png}

Bir üst yüzey var bir alt var, aralarında kalan yüzey dikey. Burada kullandığım
kavram, dikey basit bölge (vertically simple region) iki yüzey arasındaki bölge.

Başlayalım, üstteki formülün sağ ta rafındaki entegrali hesaplayalım. Bu
hesaptan bir sayı çıkmayacak tabii ki çünkü pek çok şeyi tanımsız bıraktık, ama
en azından bazı basitleştirmeler yapabiliriz, mesela bir ölçüde $z$ üzerinden
entegral alabilirim. 

$$
\iiint_D R_z \ud V = \iint \int_?^? R_z \ud z \ud x \ud y
$$

$z$'nin sınırları nedir? Hatırlarsak üçlü entegralde $z$ üzerinden entegral
alırken ise $x,y$ değişkenlerini sabitleyerek başlıyorduk, ve o sabitlenen
$x,y$'den yukarı çıkıp bir dikey kesite bakıyorduk ve sınırların nereye
geldiğini not ediyorduk. Üstteki bölge için bu altta $z_1$, üstte $z_2$. 

$$
= \iint \int_{z_1}^{z_2} R_z \ud z \ud x \ud y
$$

Şimdi tüm mümkün $x,y$ için entegralin geri kalanını hesaplamak istiyorum,
bu üstteki dikey bölgenin gölgesindeki alan $U$ içinde olacak, 

$$
= \iint_U \left( \int_{z_1(x,y)}^{z_2(x,y)} R_z \ud z  \right) \ud x \ud y  
$$

Üstteki entegrali hesaplamayı düşünürsek, en içteki entegral fazla kötü durmuyor
aslında, $R$'nin $z$'ye göre türevi var, sonra $z$ üzerinden entegral alınıyor.
Bu bize $R$'yi geri vermez mi? Evet. O zaman

$$
\iiint_D R_z \ud V = \iint_U \bigg[ R(x,y,z_2(x,y)) - R(x,y,z_1(x,y))  \bigg]
\ud x \ud y
\mlabel{2}
$$

Elde net formül olmadan daha fazla ilerleyemem, şimdi çift entegrale dönüyorum.
Bu entegralde $S$ var, ve $S$ alt, üst ve yan yüzeylerden oluşan kapalı bölge.

$$
\int \oint_{S = \textrm{alt+üst+kenarlar}} < 0,0,R > \cdot \hat{n} \ud S =
\iint_{\textrm{üst}} + \iint_{\textrm{alt}} + \iint_{\textrm{kenarlar}} 
$$

Üst yüzey ile başlayalım. Akış entegralindeki $\hat{n} \ud S$'i o yüzey
için hazırlamak lazım. İyi haber üst, alt yüzeyin $x,y$ üzerinden bir $z$
formülü var, ve bu tür formül olunca $\hat{n} \ud S$'i nasıl hesaplayacağımızı
biliyoruz, mesela $z=z_2(x,y)$ için,

$$
\hat{n} \ud S =
< -\frac{\partial z_2}{\partial x},
-\frac{\partial z_2}{\partial y},
1 >
\ud x \ud y
$$

Tabii üstteki $z_2$ üzerinden kısmi türevleri de hesaplayamıyoruz ama
iki üstteki formülde $< 0,0,R >$ ile bir noktasal çarpım var, $z_2$
türevi içeren ilk iki terim yokolacak, geriye 1 ile çarpılan $R$
kalacak,

$$
< 0,0,R > \cdot \hat{n} \ud S = R \ud x \ud y
$$

Bu demektir ki 

$$
\iint_{\textrm{üst}} < 0,0,R > \cdot \hat{n} \ud S =
\iint_{\textrm{üst}} R \ud x \ud y
$$

olur. $z$ için üst kısımdeki $z_2$ formülünü kullanırız,

$$
= \int \int_U R(x,y,z_2(x,y)) \ud x \ud y
$$

ve $z,y$ değişken sınırları için, bölgem tam $U$ üzerinde duruyor o zaman
sınırları $U$ belirler.

Alt kısım için aynı yöntem, $z = z_1(x,y)$ formülünü kullanıyoruz, ilk başta
alttakini yazabilirdik,

$$
\hat{n} \ud S =
< -\frac{\partial z_1}{\partial x},
-\frac{\partial z_1}{\partial y},
1 >
\ud x \ud y
$$

Yanlız bir noktaya dikkat, işaret yönü (orientatıon) önemli. Uzaklaşım
Teorisi'ni tanımlarken normal vektörlerin tanımladığımız bölgelerden dışarı
doğru işaret etmesi gerektiğini söylemiştik. Bu durumda alt bölgede $\hat{n}$
aşağı doğru işaret eder. Bu sebeple üstteki vektörü ters donduruyoruz,

$$
\hat{n} \ud S =
<\frac{\partial z_1}{\partial x},
\frac{\partial z_1}{\partial y},
-1 >
\ud x \ud y
$$

Devam ediyoruz, önceki noktasal çarpımı tekrarlayalım,

$$
< 0,0,R > \cdot \hat{n} \ud S = -R \ud x \ud y
$$

Entegre ediyoruz,

$$
\int \int_U -R (x,y,z_1(x,y)) \ud x \ud y
$$

Şimdi sıra bölge kenarlarına geldi. Fakat kenarların $z$ eksenine paralel,
yukarı doğru dimdik olduğunu söylemiştik, o zaman bu kenarlardan akış olamaz,
demek ki kenar bölgeler için hesaba gerek yok. Zaten baştaki problem tanımı
(1)'de bu sebeple tanımı basit tutmuştum, vektör alanını $< 0,0,R > $,
sadece $z$ bileşenini içerek şekilde tanımlamıştım ki şimdi bu ek hesaptan
kurtulabileyim.

Sonuc

(2)'de $\iiint_D R_z \ud V$ için bir formül elde etmiştik. Eğer akışı üst
ve alt bölgeler için toplarsam aynı sonuca eriseceğim.

$$
\iiint_D R_z \ud V = \int \oint_{\textrm{üst+alt+kenarlar}} < 0,0,R >
\cdot \hat{n} \ud S
$$

İspatın bu kısmını tamamlamış olduk.

[devam edecek]

\end{document}



