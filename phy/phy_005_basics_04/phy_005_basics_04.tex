\documentclass[12pt,fleqn]{article}\usepackage{../../common}
\begin{document}
Temel Fizik 4 - Atalet Matrisi (Inertia Matrix, Tensor)

Bir objenin havaya fırlatıldığını düşünelim, fırlatma sırasında dönüş te var,
çetrefil bir hareket sözkonusu yani. Fakat şimdiye kadar gördüğümüz teknikler
ile hala bu hareketi analiz edebiliriz, hem lineer momentum, hem de açısal
momentum kütle merkezi odaklı olarak analiz edilebiliyor. Herhangi bir katı
gövde, cisim şeklini ve hareketi analiz için şimdi bazı genel formülleri
ortaya koyalım. 

Gövdenin açısal momentumu $L$ hesabı için

$$
L = \sum m_i r_i \times v_i
$$

ki $L,r,v$ vektör. $v = \omega \times r$ üzerinden,

$$
= \sum m_i r_i \times (\omega \times r_i)
$$
























\end{document}
