\documentclass[12pt,fleqn]{article}\usepackage{../../common}
\begin{document}
Markov Zincirleri

Markov Zincirlerinde (MZ) $i$ konumundan $j$ konumuna geçiş olasılığını,
$P_{ij}$, biliriz ve bu $P(X_{n+1} = j | X_{n} = i)$ olarak açılabilir. Açılımdan  
görüleceği üzere bir MZ sonraki adıma geçiş olasılığı için sadece
bir önceki adıma bakar. Bu tür önce/sonra yapısındaki iki boyutlu hal, 
çok rahat bir şekilde matrisine çevirilebilir / gösterilebilir. Önceki konum 
satırlar, sonraki konum kolonlar olarak betimlenir mesela. 

Örnek

Bir sonraki günde yağmur yağmayacağını bir MZ olarak tasarlayalım [1, sf 196].
Bir sonraki günde yağmur yağmayacağını sadece bugün etkiliyor olsun. Eğer bugün
yağmur yağıyorsa yarın yağmur yağması 0.7, eğer bugün yağmıyor ise yarın
yağması 0.4. MZ şöyle

$$ 
P =
\left[\begin{array}{cc}
0.7 & 0.3 \\
0.4 & 0.6
\end{array}\right]
$$

Geçiş olasılıklarından bahsettiğimize göre ve elimizde sınırlı / belli
sayıda konum var ise, bir MZ'nin her satırındaki olasılıkların toplamı
tabii ki 1'e eşit olmalıdır. 

MZ'lerin ilginç bir özelliği $n$ adım sonra $i,j$ geçişinin $P_{i,j}^n$
hesabıyla yapılabilmesidir. Yani $P$'yi $n$ defa kendisiyle çarpıp $i,j$
kordinatına bakarsak $n$ adım sonrasını rahatça görebiliriz. İspat için
[1, sf. 195].

Mesela üstteki örnekte, eğer bugün yağmur yağıyorsa 4 gün sonra yağmur
yağma olasılığı nedir? 

\begin{minted}[fontsize=\footnotesize]{python}
import numpy.linalg as lin
P = np.array([[0.7,0.3],[0.4,0.6]])
P4 = lin.matrix_power(P,4)
print P4
\end{minted}

\begin{verbatim}
[[ 0.5749  0.4251]
 [ 0.5668  0.4332]]
\end{verbatim}

Aradığımız geçiş için kordinat 0,0'a bakıyoruz ve sonuç 0.5749. Numpy
\verb!matrix_power! bir matrisi istediğimiz kadar kendisiyle çarpmamızı
sağlıyor. 

Durağan Dağılım (Stationary Distribution)

Eğer yağmur örneğindeki matrisi çarpmaya devam edersek, mesela 8 kere
kendisiyle çarpsak sonuç ne olurdu? 

\begin{minted}[fontsize=\footnotesize]{python}
import numpy.linalg as lin
P = np.array([[0.7,0.3],[0.4,0.6]])
P8 = lin.matrix_power(P,8)
print P8
\end{minted}

\begin{verbatim}
[[ 0.57145669  0.42854331]
 [ 0.57139108  0.42860892]]
\end{verbatim}

Dikkat edilirse, her satır bir değere yaklaşmaya başladı. Bu değer MZ'nin
durağan dağılımıdır, belli koşullara uyan her MZ'nin böyle bir durağan
dağılımı vardır. Bu koşullar MZ'nin periyodik olmayan (aperiodic) ve tekrar
eden (recurrent) olmasıdır. Bu şartlar çok ``özel'' şartlar değildir
aslında, daha çok ``normal'' bir MZ'yi tarif ediyor diyebiliriz. Tüm
konumları tekrar eden yapmak kolaydır, MZ tek bağlı (singly connected) hale
getirilir, yani her konumdan her diğer konuma bir geçiş olur, ve periyodik
olmaması için ise MZ'ye olmadığı zamanlarda bir konumdan kendisine geçiş
sağlanır (az bir gürültü üzerinden). 

Nihayet durağanlık şu denklemi ortaya çıkartır,

$$ \pi = \pi P $$

Burada durağan dağılım $\pi$'dir. Bu denklem tanıdık geliyor mu?  Devriğini
alarak şöyle gösterelim, belki daha iyi tanınır, 

$$ P^T\pi^T = \pi^T $$

Bir şey daha ekleyelim, 

$$ P^T\pi^T = 1 \cdot \pi^T $$

Bu özdeğer/vektör formuna benzemiyor mu? Evet! Bu form 

$$ Ax = \lambda x $$

MZ denklemi şunu söylüyor, 1 değerindeki özdeğere ait özvektör bir MZ'nin
durağan dağılımıdır! Bu arada, MZ geçiş matrisi $P$'nin en büyük
özdeğerinin her zaman 1 olduğunu biliyoruz (çünkü üstteki tarif ettiğimiz
özel şartlara sahip olan türden matrisler böyle özdeğerlere sahip
olmalı). Bu durumda en büyük özdeğere ait özvektörü hesaplamak yeterli
olacaktır. Bunu yapmayı zaten [2]'de öğrenmiştik, üst metot (power method)
sayesinde bu hesap kolayca yapılabiliyor.



Kaynaklar

[1] Ross, {\em Introduction to Probability Models, 10th Ed}

[2] Bayramli, {\em Lineer Cebir, Ders 21}

\end{document}
