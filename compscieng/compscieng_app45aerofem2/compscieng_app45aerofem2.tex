\documentclass[12pt,fleqn]{article}\usepackage{../../common}
\begin{document}
Sıvı Dinamiğinde Sonlu Öğeler Metotu (Finite Elements Method -FEM-) - 2

FEM'e Giriş

Ağırlıklı artıklar metotuna baktık, ve Galerkin metotunun onun özel bir durumu
olduğunu gördük. Şimdi Galerkin metotu konusunda daha da ilerleyeceğiz ve daha
önce belirttiğimiz gibi bu metot sonlu öğeler analizinin temelini oluşturur.

Basit bir örnekle başlayalım. Lineer, istikrarlı, tek boyutlu bir problem, iki
boyutlu normal diferansiyel denklem (ODE). İlk önce problemin kuvvetli formuna
(strong form) bakalım. Problem şu şekilde, $f$ verili, bulmamız gereken $u$,

$$
u_{xx} + f = 0
$$

olmalı. İki sınır şartı var, biri Drichlet şartı,

$$
u(1) = g
$$

Diğeri von Neumann şartı,

$$
-u_{x}(0) = h
$$

Bu denklem mesela sıcaklık yayımında görülebiliyor, $u$ bir işi alanı (field)
olabilir, bir çubuk üzerinde $0,..,x$ noktalarındaki sıcaklık $u(x)$ olur yani,
ana denklem kalıcı durum sıcaklık iletkenliği kurallarından rahatça
türebilebilir. $f$ de bir ısı kaynağı olabilir. Ardından iki tane sınır şartı
var, $x=1$ noktasında mesela ısı $u$ değeri $g$ olmalı, bir de akış (flux) için
bir şart var, ısı alanının türevi (ki bu akış demek) belli bir $h$ değerinde
olmalı. Isısız (adiabatic) durumda $h$ tabii ki sıfır.

Problem diyor ki öyle bir $u$ bul ki ana denge formülünü tatmin etsin,
ayrıca iki sınır şartını da tatmin etsin.

Usttekileri kuvvetli form olarak kategorize ediyorum. 

[devam edecek]

Kaynaklar

[1] Mittal, {\em FEM for Fluid Dynamics, Lecture 07 Part A, Method of Weighted Residuals, IIT Kanpur},
    \url{https://www.youtube.com/channel/UCWheqBdP45xBVp_Eqi1eltQ/videos}

\end{document}
