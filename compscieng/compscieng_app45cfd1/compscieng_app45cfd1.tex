\documentclass[12pt,fleqn]{article}\usepackage{../../common}
\begin{document}
Hesapsal Sıvı Dinamiği (Computational Fluid Dynamics -CFD-) - 1

Tek boyutlu lineer taşınım akımı (convection), ya da tek boyutlu lineer yatay
iletim (advection), CFD hakkında bir şeyler öğrenmek için güzel fırsatlar
içeriyor. Bu ufak denklemin bize ne kadar çok şey öğreteceğini görmek bizi
şaşırtabilir. Denklem,

$$
\frac{\partial u}{\partial t} +
c \frac{\partial u}{\partial c}  = 0
$$

Dikkat bu bir dalga denklemi olarak bilinir, fakat esas dalga denklemini kısmı
türevsel formu ikinci kısmı türevi içeriyor, bkz [2]. Üstteki denklem verili
başlangıç şartlarına göre bir basit dalganın şekil değiştirmeden $c$ hızında
yayılmasını temsil eder. Başlangıç şartlarını $u(x,0) = u_0(x)$ olarak
gösterirsek, denklemin kesin analitik çözümü $u(x,y) = u_0(x-ct)$. 

Dalga denklemini hem zaman, hem uzay bağlamında ayrıksallaştıracağız. Türev
tanımından (ve limit ifadesini çıkartınca),

$$
\frac{\partial u}{\partial x} \approx
\frac{u(x+\Delta x) - u(x)}{\Delta x}
$$

olduğunu biliyoruz. Şimdi zamanda İleri Farklılık (Forward Difference), uzayda
Geriye Farklılık (Backward Difference) kullanalım.. Ve eğer $x$ eksenini $N$
parçaya ayırırsak ve bu parçaları $i=0,..,N$ ile indekslersek, ve en ufak zaman
adımını da $\Delta t$ ile gösterip o adımı $n$ ile indislersek,

$$
\frac{u_i^{n+1} - u_i^n}{\Delta t} + c \frac{u_i^{n} - u_{i-1}^n}{\Delta x} = 0
$$

ki $n$ ve $n+1$ ardı ardına olan iki zaman adımı, $i-1$ ve $i$ ise
ayrıksallaştırılmış iki $x$ yeri oluyor. Eğer başlangıç koşulları verilmiş ise o
zaman bu ayrıksal sistemde tek bilinmeyen $u_i^{n+1}$'dir. Denklemi tekrar
düzenlersek bilinmeyen için yeni bir formül elde edebiliriz,

$$
u_i^{n+1} = u_i^n - c \frac{\Delta t}{\Delta x} ( u_i^n - u_{i-1}^n )
$$








[devam edecek]

Kaynaklar

[1] Barba, {\em 12 steps to Navier–Stokes, Ders 1},
    \url{https://nbviewer.jupyter.org/github/barbagroup/CFDPython/blob/master/lessons/01_Step_1.ipynb}

[2] Bayramlı, {\em Kısmi Türevsel Denklemler, Dalga Denklemini Türetmek}

\end{document}
