\documentclass[12pt,fleqn]{article}\usepackage{../../common}
\begin{document}
Hesapsal Sıvı Dinamiği (Computational Fluid Dynamics -CFD-) - 2

Diffusion (Yayınım) Denklemi

Tek boyuttaki yayınım denklemi,

$$
\frac{\partial u}{\partial t} = \nu \frac{\partial^2 u}{\partial x^2}
$$

Dikkat edersek bu denklemde bir ikinci kısmı türev var. Denklemin o
kısmını Merkezi Farklar yaklaşımı ile ayrıksal hale getireceğiz, bu
yaklaşım İleri Farklar ve Geriye Farklar yaklaşımlarının birleştirilmesi ile
elde edilir.

Önce Taylor serilerini hatırlarsak, genel tanım

$$
f(x+h) = f(x) + h f'(x) + \frac{h^2}{2} f''(x) + ...
$$

Biz $u_{i+1}$ ve $u_{i-1}$ açılımını Taylor serisi ile yapmak istiyoruz, daha
önce belirttiğimiz gibi bir önceki ve sonraki $x$ değerleri $\Delta x$
uzaklığında, yani bir önceki

$$
u(x-\Delta x) = u(x) - \Delta x f'(x) + \frac{h^2}{2} u''(x) + ...
$$

İşaretin eksi olmasına dikkat, ve sonraki 

$$
u(x+\Delta x) = u(x) + \Delta x f'(x) + \frac{h^2}{2} u''(x) + ...
$$

Şimdi indisleriyle $u$ için ve [1]'deki formuyla yazalım,

$$
u_{i+1} = u_i + \Delta x \frac{\partial u}{\partial x}\bigg|_i +
\frac{\Delta x^2}{2} \frac{\partial ^2 u}{\partial x^2}\bigg|_i +
\frac{\Delta x^3}{3!} \frac{\partial ^3 u}{\partial x^3}\bigg|_i +
O(\Delta x^4)
$$

$$
u_{i-1} = u_i - \Delta x \frac{\partial u}{\partial x}\bigg|_i +
\frac{\Delta x^2}{2} \frac{\partial ^2 u}{\partial x^2}\bigg|_i -
\frac{\Delta x^3}{3!} \frac{\partial ^3 u}{\partial x^3}\bigg|_i +
O(\Delta x^4)
$$

Bir üstteki denklemin ilk hali $u_i = u_{i-1} ... $ ile ama ufak bir yer
değişimi ile görülen biçim elde edilmiş. 

Son iki formülü toplarsak bazı terimlerin ters işaretli olması sebebiyle iptal
olacağını görebiriliz. Ayrıca yaklaşık temsil açısından $O(\Delta x^4)$ ve daha
üstü kuvvetleri yok sayarsak, 

$$
u_{i+1} + u_{i-1} =
2u_i+\Delta x^2 \frac{\partial ^2 u}{\partial x^2}\bigg|_i +
O(\Delta x^4)
$$

$\frac{\partial ^2 u}{\partial x^2}\bigg|_i$ için çözersek ve tekrar düzenlersek,

$$
\frac{\partial ^2 u}{\partial x^2}=\frac{u_{i+1}-2u_{i}+u_{i-1}}{\Delta x^2} + O(\Delta x^2)
$$

$O(\Delta x^2)$ ifadesi $O(\Delta x^4)$ terimi $\Delta x^2$ ile bölününce ortaya çıktı.

Artık 1D yayınım formülünün nihai ayrıksal halini yazabiliriz,

$$
\frac{u_{i}^{n+1}-u_{i}^{n}}{\Delta t} =
\nu\frac{u_{i+1}^{n}-2u_{i}^{n}+u_{i-1}^{n}}{\Delta x^2}
$$

Daha önce olduğu gibi başlangıç koşuları tanımlı ise tek bilinmeyen
$u_{i}^{n+1}$, bu bilinmeyen eşitliğin solunda kalacak şekilde tekrar
düzenlersek,


$$
u_{i}^{n+1} =
u_{i}^{n}+\frac{\nu\Delta t}{\Delta x^2}(u_{i+1}^{n}-2u_{i}^{n}+u_{i-1}^{n})
$$


Kaynaklar

[1] Barba, {\em 12 steps to Navier–Stokes, Ders 1},
    \url{https://nbviewer.jupyter.org/github/barbagroup/CFDPython/blob/master/lessons/01_Step_1.ipynb}




\end{document}
