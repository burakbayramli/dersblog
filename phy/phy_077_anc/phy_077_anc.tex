\documentclass[12pt,fleqn]{article}\usepackage{../../common}
\begin{document}
Modern Bilim Öncesi Bilimsel, Astronomik Buluşlar, Gezegenler, Yörüngeler

Dünya - Ay Mesafe Oranı

Aristarchus dünya-güneş ve dünya-ay mesafeleri arasında bir oran hesaplamayı
başardı. Bunu yapmak için basit açılar kullanması yeterli oldu. Önce ayın
yarım ay fazına gelmesini bekledi, 

\includegraphics[width=20em]{moonshad.jpg}

Bu durumda güneş-ay-dünyanin birbirine belli bir şekilde durması doğaldır, ki
bu ilişki alttaki gibi çizilebilir,

\includegraphics[width=20em]{sunmoon.png}

Acı $\varphi$ dünyadaki aletlerle ölçülebilir, kabaca aya doğru kolumuzu dik
uzatırız, sonra dondurup güneye doğru işaret ettik diyelim, bu geçişin açısı
$\varphi$ açısıdir. Bu açıyi Aristarchus 97 derece olarak ölçtü. Dünya-Güney-Ay
üçlüsünün o andaki yerlerinin bir dik üçgenin köşeleri olduğunu biliyordu, çünkü
yarım ay fazından bu böyle olmalıydı. O zaman $\varphi$ biliniyor, $S/L$ oranı
nasıl bulunur? $\varphi$ için kosinüs hesabı $L/S$ değil midir?  Evet. Onu ters
çevirirsek, istediğimiz orana erişiriz, $1 / \cos\varphi$,

\begin{minted}[fontsize=\footnotesize]{python}
print (1./np.cos(np.deg2rad(87)))
\end{minted}

\begin{verbatim}
19.10732260929735
\end{verbatim}

Yani güneş bize aydan yaklaşık 19 kat daha uzaktadır. Bunu sadece basit açılarla
hesaplayabilmiş olduk.

Dünyanin Yuvarlaklığı, ve Çevre Uzunluğu

Eratosten (Erotosthenes) MO 276 - MO 194 yıllarında yaşayan bilim adamıdır,
İskenderiye kütüphanesinin başındaydı. Bir gün birinden öğrendi ki yazın en uzun
günü 21. Haziran'da daha güneyde olan Syene şehrinde eğer yere bir çubuk
dikilirse, saat 12'de çubuğun hiç gölgesi olmuyor. Pitagor zamanından beri
aslında dünyanin yuvarlak olabileceği düşünülüyordu, Eratosten acaba aynı uzun
yaz gününde daha kuzeyde olan İskendiriye'de yere bir çubuk dikersem saat 12'de
ne görürüm.

Bunu yaptı ve gördü ki ufak ta olsa bir gölge var. 

\includegraphics[width=20em]{circum3.jpg}

Sonra bu gölgenin sonuna kadar sopa basından doğru bir çizgi çekince, oluşan
açıya baktı,

\includegraphics[width=20em]{circum4.jpg}

Bu önemli bir bilgiydi çünkü şimdi gökyüzüne düşen güneş ışınlarını düşünelim,
işin öyle düşüyor ki o açı oluşmuş, 

\includegraphics[width=20em]{circum1.jpg}

Şimdi bu ışını düz uzatırsak, bir de sopanın yönünde direk bir çizgiyi direk
dünya merkezine çekersek, bir üçgen ortaya çıkar, bir çizgiyi Syene şehrinin
sopasından direk dünya merkezine uzatabiliriz (bu çizgi direk merkeze gider
çünkü biliyoruz ki o anda oradaki sopanın gölgesi yok, güneş ışını direk sopanın
tepesine geliyor)

\includegraphics[width=20em]{circum2.jpg}



Kaynaklar

[1] Wikipedia,
    \url{https://en.wikipedia.org/wiki/On_the_Sizes_and_Distances_(Aristarchus)}

[2] Science Insider, {\em How The Ancient Greeks Proved Earth Wasn't Flat 2,200 Years Ago},
    \url{https://youtu.be/EfZ2HZH5CkA}
    
\end{document}
