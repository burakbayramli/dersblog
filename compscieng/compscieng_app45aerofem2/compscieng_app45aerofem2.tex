\documentclass[12pt,fleqn]{article}\usepackage{../../common}
\begin{document}
FEM'e Giriş

Ağırlıklı artıklar metotuna baktık, ve Galerkin metotunun onun özel bir durumu
olduğunu gördük. Şimdi Galerkin metotu konusunda daha da ilerleyeceğiz ve daha
önce belirttiğimiz gibi bu metot sonlu öğeler analizinin temelini oluşturur.

Basit bir örnekle başlayalım. Lineer, istikrarlı, tek boyutlu bir problem, iki
boyutlu normal diferansiyel denklem (ODE).

İlk önce problemin kuvvetli formuna (strong form) bakalım.



Kaynaklar

[1] Mittal, {\em FEM for Fluid Dynamics, Lecture 07 Part A, Method of Weighted Residuals, IIT Kanpur},
    \url{https://www.youtube.com/channel/UCWheqBdP45xBVp_Eqi1eltQ/videos}

\end{document}
