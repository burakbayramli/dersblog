\documentclass[12pt,fleqn]{article}\usepackage{../../common}
\begin{document}
Materyel Mekaniği - 2

Sonsuz Küçük (Infinitesimal) Gerginlik Tensörü

Green gerginlik tensorunu gördük, kuvvetli bir yaklaşım ama bizim daha çok
kullanacağımız şimdi anlatacağımız. Niye? Çünkü Green tensoru sonlu değişimler
için geçerli ama çoğu uygulamada bize lazım olan çok ufak yamulmalar. Ufak
değişimler derken, (3) formülünden hareketle, oradaki en son terimi hatırlarsak,
çok ufak yamulmalar için $\nabla u^T \nabla u << \nabla u$ olur, yani ufak
değişimlerde o karesel işlem $\nabla u$'dan daha ufak sonuç verir. O zaman belli
durumlarda son terim yaklaşık sıfır kabul edilebilir, $\nabla u^T \nabla u
\approx 0$










[devam edecek]

\end{document}
