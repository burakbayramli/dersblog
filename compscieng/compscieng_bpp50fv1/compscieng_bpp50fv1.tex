\documentclass[12pt,fleqn]{article}\usepackage{../../common}
\begin{document}
Sonlu Hacim (Finite Volume) Yöntemi - 1

Üç boyutlu kütle muhafazası üzerinden süreklilik formül [2]'de işlendi.  Şimdi
tek boyutlu ortamda muhafaza kanunlarını işleyeceğiz, gaz dinamiği, genel
aerodinamik konularında bu yaklaşım faydalı olacak. Sayısal çözmeye çalışılacak
problemler, ki sonlu hacim (finite volume -FV-) yöntemi burada lazım, muhafaza
kanunları içeren hiperbolik sistemleridir (hyperbolic systems of conservation
laws). Bu tür sistemler zamana bağlı çoğunlukla gayrı lineer kısmi türevsel
denklemlerdir (nonlinear PDE), ve aslında basit yapıları vardır. Tek yersel
boyutta şuna benzerler [3, sf. 1],

$$
\frac{\partial }{\partial t} u(x,t) + 
\frac{\partial }{\partial x} f(u(x,t)) = 0
\mlabel{1}
$$

Daha önce [1]'de Burgers'in denklemini görmüştük, bir PDE,

$$
u_t + uu_x = 0
\mlabel{2a}
$$

Bu denklem (1) ışığında düşünülebilir, eğer $f(u) = u^2/2$ tanımlarsak,
(1) formülü, yani $u_t + f(u)_x = 0$, formül (2a) ile aynıdır. O zaman,

$$
u_t + f(u)_x = 0, \qquad f(u) = \frac{1}{2}u^2
\mlabel{2b}
$$

(2b)'ye erişmek mümkün, şöyle, $t$ ve $x$ değişkenleri üzerinden ve $x = x(t)$
olduğunu hatırlayalım, yani $u = u(x,t) = u(x(t),t)$ olur.

İki değişkenli fonksionlar üzerinde genel zincirleme kanununu hatırlayalım [10]
mesela $z = f(x(t),y(t))$ için $\ud z / \ud t$

$$
\frac{\ud z}{\ud t} =
\frac{\partial z}{\partial x} \cdot \frac{\ud x}{\ud t} +
\frac{\partial z}{\partial y} \cdot \frac{\ud y}{\ud t} 
$$


Simdi iki degiskenli ve $u$'nun zamana göre türevi,

$$
\frac{\ud}{\ud t} u (x(t),t)) = \frac{\partial }{\partial t} u(x(t),t) 
$$






(2) türünden denklemleri tek boyutta çözmeyi işleyeceğiz öncelikle, çünkü çok
boyutta çözüm tek boyuta indirgenerek yapılabiliyor.

Hiperbolik denklemleri analitik, kesin (exact) çözmek için birkaç konuyu
yakından anlamak lazım. Birincisi Riemann problemleri; bu yaklaşımla hiperbolik
PDE'nin başlangıç koşulu kesintili (discontinuous) bir fonksiyonla belirtiliyor
ve bu çözümleri çoğu durumda daha rahatlaştırılıyor, diğeri hiperbolik muhafaza
kanunlarının entegral formu.

İleride hiperbolik denklemleri FV ile sayısal çözerken de Riemann yaklaşımı
faydalı olacak. Kesintili başlangıç içeren denklemler çözebilmek önemli çünkü FV
ile sayısal çözüm yaparken uzayı parçalara bölüyoruz, ve her iki parçayı bir
kesintili başlangıç içeren Riemann problemi olarak temsil ediyoruz, bu pek çok
parça ortaya çıkartır tabii, bu sebeple tipik bir FV yaklaşımı her adımda pek
çok Riemann problemini çözecektir.

Entegral form lazım, çünkü sınırlı farklılıklarda (finite difference) olduğu
gibi ayrıksal olan fonksiyonun eşit aralıklarda tanımlı bir ızgaranın seçilmiş
belli noktaları değil, her bölge, parçanın ortalaması, yani entegrali.

Entegral form ile başlayalım. Aslında diferansiyel form entegral formden
türetilmiştir -bu türetim pürüzsüzlük faraziyesi üzerinden
yapılmıştır-. Özellikle kesintili başlangıç şartları olduğu durumlarda
diferansiyel formun her yerde düzgün işlemesi mümkün değil, çünkü kesintilerde
türev alınamıyor. Ayrıca pür kesintisiz olsa bile şok oluşumu denen sebeplerle
türevsel fonksiyonlar çözülemiyor. Bu problemlerle başedebilmek için entegral
formu kullanmak gerekecek.

Formu yaratalım. İçinde gaz olan sadece tek boyutuna baktığımız bir tüp
düşünelim, $x$ tüpün üzerindeki bir noktayı temsil edecek, $\rho(x,t)$ ise tüpün
$x$ noktasında ve $t$ anındaki yoğunluğunu verecek diyelim. Yoğunluğu kullanarak
$x_1$ ve $x_2$ noktaları arasındaki $t$ anındaki kütle

$$
\int _{x_1}^{x_2} \rho(x,t) \ud x
$$

ile hesaplanabilir. Tüpün duvarları tam izole ise ve kütle yoktan varedilip
yokedilemeyeceğine göre tüpe gaz giriş ya da çıkış sadece $x_1,x_2$
noktalarından olabilir [3, sf. 14]. Şimdi bir gaz hareket hızı düşünelim,
$v(x,t)$ ile, o zaman gaz akma oranı, ya da akış (flux)

$$
flux = \rho(x,t) v(x,t)
$$

olur. Üstteki fiziksel kurallardan hareketle $[x_1,x_2]$ deki kütlenin
değişim oranı $x_1$ ve $x_2$ noktalarındaki akışın farkına eşit olmalıdır,

$$
\frac{\ud}{\ud t} \int _{x_1}^{x_2} \rho(x,t) \ud x =
\rho(x_1,t) v(x_1,t) - \rho(x_2,t) v(x_2,t)
$$

İşte bu muhafaza kanununun entegral formudur. 

Üstteki formülü $t_1,t_2$ zaman aralığı için entegre edersek, ki böylece
bu zaman içindeki tüm toplam akışı hesaplayabilelim, o zaman

$$
\int_{t_1}^{t_2} \left( \frac{\ud}{\ud t} \int _{x_1}^{x_2} \rho(x,t) \ud x  \right)  =
\int_{t_1}^{t_2} \rho(x_1,t) v(x_1,t) \ud t -
\int_{t_1}^{t_2} \rho(x_2,t) v(x_2,t) \ud t
$$

Soldaki kısım zaman üzerinden türevin yine zaman üzerinden entegrali, o zaman
yokolabilir, Calculus'un Temel Teorisi üzerinden basitleştirirsek,

$$
\int _{x_1}^{x_2} \rho(x,t_2) \ud x -
\int _{x_1}^{x_2} \rho(x,t_1) \ud x  = 
\int_{t_1}^{t_2} \rho(x_1,t) v(x_1,t) \ud t -
\int_{t_1}^{t_2}  \rho(x_2,t) v(x_2,t) \ud t
$$

Ufak bir yer değiştirme sonrası

$$
\int _{x_1}^{x_2} \rho(x,t_2) \ud x =
\int _{x_1}^{x_2} \rho(x,t_1) \ud x  +
\int_{t_1}^{t_2} \rho(x_1,t) v(x_1,t) \ud t -
\int_{t_1}^{t_2}  \rho(x_2,t) v(x_2,t) \ud t
$$

Üstteki formun değişik bir şekli ileride lazım olacak, zaman adımı atmaya
uğraştığımız hesapsal yöntemlerde $t_1$ ve $t_2$ üzerinden bir entegral, hesabı
bir sonraki zamana geçirmeye uğraştığımızda, adım attığımızda.

Neyse şimdi diferansiyel forma geçise dönelim. Bu noktada $\rho(x,t)$ ve
$v(x,t)$'nin türevi alınabilir fonksiyonlar olduğunu farz ediyoruz. Üstekini,
yine ufak bir değişim sonrası,

$$
\int _{x_1}^{x_2} \rho(x,t_1) \ud x  +
\int _{x_1}^{x_2} \rho(x,t_2) \ud x -
\int_{t_1}^{t_2}  \rho(x_2,t) v(x_2,t) \ud t -
\int_{t_1}^{t_2} \rho(x_1,t) v(x_1,t) \ud t = 0
\mlabel{3}
$$

olarak görelim. Eğer Calculus'un Temel Teorisi ile ilk iki terime
$\int_{t_1}^{t_2} .. \ud / \ud t$ son iki terime $\int_{x_1}^{x_2} .. \ud / \ud x$
ekleyebilirsek, tüm terimlerde aynı entegraller olacağı için, 
$\int_{t_1}^{t_2} \int_{x_1}^{x_2} $ altında tüm terimleri gruplayıp
basitleştirmek mümkün, ve bunlar sıfıra eşit olur. Bu bizi diferansiyel
forma götürebilir. Yani

$$
\rho(x,t_2) - \rho(x,t_1) = \int_{t_1}^{t_2}
\frac{\partial }{\partial t} \rho(x,t) \ud t
$$

ve

$$
\rho(x_2,t)v(x_2,t) - \rho(x_1,t)v(x_1,t) =
\int _{x_1}^{x_2} \frac{\partial }{\partial x} (\rho(x,t)v(x,t)) \ud x
$$

eşitliklerinden hareketle, bunları (3)'e uygulayıp

$$
\int _{t_1}^{t_2} \int _{x_1}^{x_2}  \left\{
\frac{\partial }{\partial t} \rho(x,t)  +
\frac{\partial }{\partial x} (\rho(x,t)v(x,t))
\right\} \ud x \ud t = 0
\mlabel{4}
$$

elde ediyoruz. Bu ifadenin $[x_1,x_2]$ ve $[t_1,t_2]$ arasındaki tüm değerlerde
doğru olması gerektiği için entegre edilenin sıfır olması gerekiyor ([2]'dekine
benzer bir mantık yürütüldü), yani

$$
\rho_t + (\rho v)_x = 0
$$

olmalı. Böylece kütlenin muhafaza kuralını diferansiyel formda elde etmiş olduk.

Bu formu izole halde çözmenin tek yolu $v$'nin önceden bilindiği durumdadır, ya
da $v$ fonksiyon $\rho(x,t)$'ye bağlı bir fonksiyon olmalıdır, yani
$f(\rho) = \rho v$ gibi. Bu durumda üstteki ifade $\rho$ için tek sayısal
muhafaza kanunu haline gelir,

$$
\rho_t + f(\rho)_x = 0
$$


Riemann Problemi

Kesintili ve iki parça içeren bir fonksiyon ile Burgers denkleminin çözümü
mümkün; bu aslında basit, $u_t + u u_x = 0$ denklemi için başlangıç şartları

$$
u(x,0) = 
\left\{ \begin{array}{ll}
u_l & x < 0 \\
u_r & x > 0 
\end{array} \right.
\mlabel{9}
$$

olduğu durumda çözüm özgün bir zayıf çözümdür, eğer $u_l > u_r$ ise (bu mümkün
seçeneklerden birincisi)

$$
u(x,t) = 
\left\{ \begin{array}{ll}
u_l & x < st \\
u_r & x > st 
\end{array} \right.
$$

ki $s$ şok hızıdır. Ya da

$$
u(x,t) = 
\left\{ \begin{array}{ll}
u_l & x/t < s \\
u_r & x/t > s 
\end{array} \right.
$$

Kesinti noktası $s$ hızında sağa ilerler, $t$ anında olacağı yer $st$'dir.

\includegraphics[width=20em]{compscieng_bpp50fv1_01.png}

Karakteristik

\includegraphics[width=20em]{compscieng_bpp50fv1_03.png}

\includegraphics[width=10em]{compscieng_bpp50fv1_04.png}


Daha önce tek boyutlu lineer taşınım akımı (convection) ile gördüğümüz durum
burada da var, orada çözüm $u(x,y) = u_0(x-ct)$ idi, dalga hızı $c$. Şimdi hız
$u$ bu $s$ şok hızınını verir, Burgers için hesabı $s = (u_l + u_r) / 2$.  Şok
hızının hesabı için kesinti bölgesinin yeterince uzağında $M$ ve $-M$ noktalarını
seçelim, bu iki nokta arasındaki toplam kütlenin / dalganın değişiminin hızı şok
hızı $s$ olacaktır.

$$
\frac{\ud}{\ud t} \int_{-M}^{M} u(x,t) \ud x = f(u_l) - f(u_r)
\mlabel{8}
$$

Salt entegralin nasıl hesaplanacağına bakarsak [3, sf. 31],

$$
\int_{-M}^{M} u(x,t) \ud x =
\int_{-M}^{st} u_l \ud x + 
\int_{st}^{M} u_r \ud x
$$

$$
= (M+st)u_l + (M-st)u_r
$$

Şimdi zaman türevini geri koyalım, bu sağ tarafta $s(u_l-u_r)$ verir, hepsi
bir arada,

$$
\frac{\ud}{\ud t} \int_{-M}^{M} u(x,t) \ud x = s(u_l-u_r)
$$

(8)'in sağ tarafını üstteki formüle koyunca,

$$
f(u_l) - f(u_r) = s(u_l-u_r)
$$

$$
s = \frac{f(u_l) - f(u_r)}{u_l-u_r}
$$

Böylece genel bir ifade elde ettik. Burgers denklemi özelinde,
$f(u) = u^2 / 2$ olduğuna göre,

$$
f(u_l) - f(u_r) = \frac{1}{2} u_l^2 -  \frac{1}{2} u_r^2
$$

O zaman

$$
\frac{1}{2} (u_l + u_r)(u_l - u_r) = s(u_l-u_r)
$$

diyebiliriz [5, sf. 46], basitleştirince,

$$
s = \frac{1}{2} (u_l + u_r)
$$

İkinci seçenek, seyreltilmiş dalga sonucu, bu zayıf çözüm başlangıçta
$u_l < u_r$ olduğu zaman ortaya çıkıyor,

$$
u(x,t) =
\left\{ \begin{array}{ll}
u_l & x < u_l t  \\
x/t & u_l t \le x \le u_r t \\
u_r & x > u_r t
\end{array} \right.
$$

Sağ taraf yine daha önce olduğu gibi şu hale çevirilebilir (ki birazdan
görülecek kodu anlamak için de bu form faydalı)

$$
u(x,t) =
\left\{ \begin{array}{ll}
u_l & x/t < u_l \\
x/t & u_l \le x/t \le u_r  \\
u_r & x/t > u_r 
\end{array} \right. 
$$

\includegraphics[width=20em]{compscieng_bpp50fv1_02.png}

Çözümün Burgers denklemi için doğru olduğunun sağlamasını yapmak zor değil
[9, sf. 34], mesela orta şart $u_l \le x/t \le u_r $ kısmına bakalım,
bu çözümü (2a)'ya sokarsak,

$$
\frac{\partial u}{\partial t} + u \frac{\partial u}{\partial x} =
\frac{\partial }{\partial t} \left( \frac{x}{t}  \right) +
\frac{x}{t} \frac{\partial }{\partial x} \left( \frac{x}{t}  \right) =
-\frac{x}{t^2} + \frac{x}{t} \frac{1}{t} = 0
$$

İlk ve üçüncü şartın çözüm olduğu bariz çünkü sabit sayılar, ve türevleri
alınırken sıfırlanacaklar.

\begin{minted}[fontsize=\footnotesize]{python}
def qf(q): return 0.5*q*q
    
def exact_riemann_solution(xi,u_l,u_r):
    # Shock wave
    if u_l > u_r: 
        shock_speed = (qf(u_l)-qf(u_r))/(u_l-u_r)
        q = (xi < shock_speed)*u_l \
          + (xi >=shock_speed)*u_r
        return q
    # Rarefaction wave
    else:  
        q = (xi<=u_l)*u_l \
          + (xi>=u_r)*u_r \
          + (u_l<xi)*(xi<u_r)*xi
        return q
\end{minted}

\begin{minted}[fontsize=\footnotesize]{python}
def shock():
    u_l, u_r = 5.0, 1.0

    for i,t in enumerate(np.linspace(0,1,6)):
        outfile = 'rieout/shock-%02d.png' % i
        fig, ax = plt.subplots(figsize=(5, 3))
                    
        x = np.linspace(-4, 4, 1000)
        
        q = np.array([exact_riemann_solution(xi/(t+1e-10),u_l,u_r) for xi in x])

        ax.set_xlim(-4,4)

        ax.plot(x,q,'-k',lw=2)

        ax.set_title('t=%f' % t)
    
        plt.savefig(outfile)

shock() 
\end{minted}


\includegraphics[width=20em]{rieout/shock-00.png}

\includegraphics[width=20em]{rieout/shock-04.png}


\begin{minted}[fontsize=\footnotesize]{python}        
def rarefaction():
    u_l, u_r = 2.0, 4.0
    
    for i,t in enumerate(np.linspace(0,1,6)):
        outfile = 'rieout/rarefaction-%02d.png' % (t*10)

        fig, ax = plt.subplots(figsize=(5, 3))
                    
        x = np.linspace(-4, 4, 1000)
        
        q = np.array([exact_riemann_solution(xi/(t+1e-10),u_l,u_r) for xi in x])

        ax.set_xlim(-4,4)

        ax.plot(x,q,'-k',lw=2)
    
        ax.set_title('t=%f' % t)

        plt.savefig(outfile)

rarefaction()
\end{minted}


\includegraphics[width=20em]{rieout/rarefaction-02.png}

\includegraphics[width=20em]{rieout/rarefaction-06.png}

Animasyon olarak

\begin{minted}[fontsize=\footnotesize]{python}
! convert -delay 20 -loop 0 rieout/shock*.png shock.gif
\end{minted}

\begin{minted}[fontsize=\footnotesize]{python}
! convert -delay 20 -loop 0 rieout/rare*.png rarefaction.gif
\end{minted}

Animasyon sonuç dosyaları [7] ve [8]'de bulunabilir.

Kaynaklar

[1] Bayramlı, {\em Hesapsal Bilim, Hesapsal Sıvı Dinamiğine Giriş}

[2] Bayramlı, {\em Fizik, Sıvılar, 1}

[3] Leveque, {\em Numerical Methods for Conservation Laws}

[5] Cooper, {\em Introduction to PDEs with Matlab}

[7] Bayramlı, {\em Animasyon, Şok Dalgası},
    \url{https://github.com/burakbayramli/classnotes/raw/master/compscieng/compscieng_bpp50fv1/shock.gif}

[8] Bayramlı, {\em Animasyon, Seyrelen (Rarefaction) Dalga}
    \url{https://github.com/burakbayramli/classnotes/raw/master/compscieng/compscieng_bpp50fv1/rarefaction.gif}

[9] Lee, {\em AM 260, Computational Fluid Dynamics},
    \url{https://users.soe.ucsc.edu/~dongwook/wp-content/uploads/2021/am260/html/}

[10] Bayramli, {\em Cok Degiskenli Calculus, Ders 11}
    
\end{document}
