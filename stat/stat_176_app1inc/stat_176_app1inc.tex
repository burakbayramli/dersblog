\documentclass[12pt,fleqn]{article}\usepackage{../../common}
\begin{document}
Azar Azar Istatistik (Incremental Statistics)

Artımsal (Incremental) Varyans Hesabı

[1]'de gordugumuz varyans formulunu tekrar yazarsak,

$$ = \sum _{i=1}^{n} y_i^2 - \frac{1}{n} \bigg( \sum _{i=1}^{n} y_i \bigg)^2 $$

(5) formülünü $x$ kullanarak bir daha yazalım,

$$ 
S = \sum _{i=1}^{n} x_i^2 - \frac{1}{n} \bigg( \sum _{i=1}^{n} x_i \bigg)^2  
$$

Bu formülü her yeni veri geldikçe eldeki mevcut varyansı ``güncelleme''
amaçlı olarak tekrar düzenleyebilirdik, böylece veri üzerinden bir kez
geçmekle kalmayıp en son bakılan veriye göre en son varyansı
hesaplayabilmiş olurduk. Ortalama için mesela her yeni veri bir toplama
eklenebilir, ayrıca kaç veri noktası görüldüğü hatırlanır, ve o andaki en
son ortalama en son toplam bölü bu en son sayıdır. 

Fakat varyans için (5)'in bir problemi var, $\sum x_i^2$ ve $(\sum x_i)^2$
sayıları uygulamalarda aşırı büyüyorlar, ve yuvarlama hataları (rounding
errors) hataları ortaya çıkmaya başlıyor. Eğer varyans küçük ise bu aşırı
büyük sayılardaki tüm basamaklar birbirini iptal eder, geriye hiçbir şey
kalmaz. Bu hatalardan uzak durmak için varyansı farklı bir artımsal
yöntemle hesaplamak istiyoruz.

Youngs ve Cramer'in yöntemine göre [3, sf. 69] bu hesap şöyle
yapılabilir. $T_{ij}$, $M_{ij}$ ve $S_{ij}$,  veri noktaları $x_i$ $x_j$
arasındaki verileri kapsayacak şekilde sırasıyla toplam, ortalama ve
verinin karesinin toplamı olsun, 

$$ 
T_{ij} = \sum _{k=i}^{j} x_k , \quad  
M_{ij} = \frac{1}{(j-1+1)}, \quad
S_{ij} = \sum _{k=i}^{j} (x_k - M_{ij})^2
$$

Güncelleme formülleri şunlardır, 

$$ T_{1,j} = T_{i,j-1} + x_j$$

$$ S_{1,j} = S_{i,j-1} + \frac{1}{j(j-1)} (jx_j - T_{1,j})^2  $$

ki $T_{1,1} = x_1$ ve $S_{1,1}=0$ olacak şekilde.

İspat

$$ 
\sum _{k=1}^{j} \bigg( x_k - \frac{1}{j} T_{1j} \bigg) = 
\sum _{k=1}^{j} \bigg( x_k - \frac{1}{j} (T_{1,j-1}+x_j)  \bigg)^2
$$

$$ = \sum _{k=1}^{j} \bigg(
\bigg(x_k - \frac{1}{j-1}T_{1,j-1} \bigg) + 
\bigg( \frac{1}{j(j-1)} T_{1,j-1} - \frac{1}{j} x_j\bigg) 
\bigg)^2
$$

çünkü $\frac{1}{j} = \frac{1}{j-1}-\frac{1}{j(j-1)}$


$$
= \sum _{k=1}^{j-1} \bigg( x_k - \frac{1}{j-1} T_{1,j-1} \bigg)^2  
 \bigg( x_j - \frac{1}{j-1} T_{1,j-1} \bigg)^2 +
$$
$$
2 \sum _{k=1}^{j}  \bigg( x_k - \frac{1}{j-1} T_{1,j-1} \bigg)
\bigg( \frac{1}{j(j-1)} T_{1,j-1} - \frac{1}{j} x_j \bigg) +
$$
$$
j \bigg( \frac{1}{j(j-1)} T_{1,j-1} - \frac{1}{j} x_j \bigg) 
$$

$$ 
= \sum _{k=1}^{j-1} \bigg( x_k - \frac{1}{j-1} T_{1,j-1} \bigg)^2 + 
\bigg( x_j - \frac{1}{j-1} T_{1,j-1} \bigg)^2 \bigg( 1-\frac{2}{j} \bigg) + 
j \bigg( \frac{1}{j(j-1)} T_{1,j-1} - \frac{1}{j}x_j \bigg)^2
$$

çünkü $\sum _{k=1}^{j-1} (x_k-\frac{1}{j-1} T_{1,j-1} )=0$

$$ 
= S_{1,j-1}  + \bigg( x_j - \frac{1}{j-1} (T_{1j}-x_j) \bigg) ^2
\bigg( 1-\frac{2}{j}+\frac{1}{j}\bigg)
$$

$$ = S_{1,j-1} + \frac{1}{(j-1)^2} (jx_j - T_{1j})^2 \frac{j-1}{j} $$

Bu algoritma (5) algoritmasından daha stabil. Kod üzerinde görelim,

\begin{minted}[fontsize=\footnotesize]{python}
def incremental_mean_and_var(x, last_sum, last_var, j):
    new_sum = last_sum + x
    new_var = last_var + (1./(j*(j-1))) * (j*x - new_sum)**2 
    return new_sum, new_var

N = 10
arr = np.array(range(N)) # basit veri, 0..N-1 arasi sayilar
print arr
last_sum = arr[0]; last_var = 0.
for j in range(2,N+1):
    last_sum,last_var = incremental_mean_and_var(arr[j-1], last_sum, last_var, j)

print 'YC =', last_var / N, 'Standart = ', arr.var()
print last_sum, arr.sum()
\end{minted}

\begin{verbatim}
[0 1 2 3 4 5 6 7 8 9]
YC = 8.25 Standart =  8.25
45 45
\end{verbatim}


Kaynaklar

[1] Bayramli, {\em Istatistik, Beklenti, Varyans, Kovaryans ve Korelasyon}

\end{document}
