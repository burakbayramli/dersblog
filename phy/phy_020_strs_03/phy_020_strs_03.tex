\documentclass[12pt,fleqn]{article}\usepackage{../../common}
\begin{document}
Materyel Mekaniği - 3

İlerlemeden önce birazdan lazım olacak iki formülü bulalım. Daha önce gördük ki

$$
I = \int \int X_2^2 \ud X_2 \ud X_3
$$

ve

$$
M = \int \int -X_2 \sigma_{11} \ud X_2 \ud X_3
$$

Üstteki iki formülü birleştirelim, $M$ içinde $I$ oluşturalım, ve yerine koyalım,

$$
M = \int \int -\frac{X_2}{X_2^2} X_2^2 \sigma_{11} \ud X_2 \ud X_3 =
-\frac{1}{X_2} \sigma_{11} I 
$$

Buradan $\sigma_{11}$ eşitliğine geçilir,

$$
\implies \sigma_{11} = -\frac{M X_2}{I}
$$



























Euler-Bernoulli çubuklarını temel alan analizleri üç adıma bölmek mümkündür.

1) Uygulanan yük $q$'yu kullanarak saptırma (deflection) fonksiyonu $y$'yi hesapla,

$$
q = E I \frac{\ud^4 y}{\ud X_1^4}
$$

Formülde görüyoruz eğer $q$ biliniyorsa ve elde yeterli sınır şartları var ise
(dört tane) diferansiyel denklemi kullanarak $y$'yi bulabiliriz
[1, Lecture 2, 2:02:00]. 

2) Saptırma $y$ bulunduktan sonra onu kullanarak kaykılma (shear) ve bükülme
momentini hesapla,

$$
M = E I \frac{\ud^2 y}{\ud X_1^2}, \quad
V = E I \frac{\ud^3 y}{\ud X_1^3} 
$$

çünkü sonuçta $M,V$ hesapları $y$'nin birer fonksiyonu, elde edilen $M,V$
sonuçları $X_1$'in fonksiyonları olacak tabii ki.






[devam edecek]

Kaynaklar

[1] Petitt, {\em Intro to the Finite Element Method}, University of Alberta,
    \url{https://www.youtube.com/watch?v=2iUnfPRk6Ro&list=PLLSzlda_AXa3yQEJAb5JcmsVDy9i9K_fi}



\end{document}
