\documentclass[12pt,fleqn]{article}\usepackage{../../common}
\begin{document}
Vektör Alanları ve Hesaplar

Entegre Edilmiş Kinetik Enerji (İntegrated Kinetic Enerji)

Bir kasırganın tahrip edici kuvveti nedir? Katrina, İvan, Ian gibi kasırgalar 1
ila 5 arası sayılar ile kategorize ediliyorlar. Bu sayılar Saffir-Simpson
skalasıyla, ölçüm sistemiyle alakalı, bu sisteme göre fırtanın bir dakika
içindeki en yüksek ani rüzgarı (gust) ölçülür, ve tüm fırtına bu ölçüme göre
kategorize edilir [1].

Bu sayının problemi kasırgayı sadece varabildiği en yüksek rüzgar hızı üzerinden
ölçmesi. Bu en yüksek hızın ölçülmesinin teknik olarak çıkarttığı problemler bir
yana, bu sayı bize fırtınanın kapladığı alan ve bu alan içinde rüzgar şiddetinin
nasıl dağıldığı hakkında hiçbir şey söylemiyor.

Camille ve Katrina örneklerini düşünelim, birincisinde şiddetli rüzgarlar var
ama ufak alana odaklı, ikincisinin en yüksek rüzgar hızı daha az olmasına rağmen
daha geniş alana yayılı ve SS skalasında daha küçük bir kasırga olarak geçiyor.
Fakat Katrinanın çok daha zarar verici olduğunu biliyoruz.

Acaba daha iyi bir olcum olamaz mi? Kasirgalar tehlikelidir cunku ittikleri,
hareket ettirdikleri hava bloklarinda kinetik enerji vardir. Daha az yogun olsa
da havanin bir kutlesi var, gunluk hayatta fazla dusunmesek bile bu kutle yeteri
kadar hiza ulastiginda etraftaki nesnelere carpip onlari darmadigin
edebiliyorlar, agaclar, binalar, ve bunu yaparken bir enerji transferi
gerceklestirmis oluyorlar.

Bazi bilimciler bu sebeple SS skalasi yerine IKE adli bir hesabi tercih
ediyorlar. Bu hesap

$$
IKE = \int_v \frac{1}{2} \rho U^2 dV
$$

ile yapılır, $v$ hacim, $\rho$ yoğunluk, $U$ ise hızdır. Aslında burada yapılan
standart $1/2 m v^2$ hesabının bir çeşidi (son $v$ hız, hacim değil).

Sayısal bağlamda elimizde kutular olacak, her kutu içindeki hava miktarı belli
olur, onun kütlesini referans alabiliriz. Kütle hesabı için aslında tek alan
hesabı yeterli olacak çünkü hava yoğunluğu olarak 1 $kg/m^3$ farzedeceğiz, kutu
yüksekliği olarak 1 metre, böylece kutu alanı hesabı sonrası çarpı 1 $kg/m^3$ ve
çarpı 1 metre ile aynı sayıdır, kütleyi direk alandan elde etmiş oluruz.

Her kutu içindeki rüzgar hızı yatay ve dikey bileşenleri $u,v$ ile gelecek, $hız
= \sqrt{u^2+v^2}$ ile hız hesaplanabilir ya da, nasıl olsa hız karesi enerji
için lazım, $u^2+v^2$ yeterli. 0.5 çarpı hız karesi çarpı üstte bahsedilen
kütleyi çarpıp bunu kasırganın etkili olduğu coğrafyadaki tüm kutular için
yapıp toplarsak kasırga İKE'sini elde etmiş oluruz 

kenari 1400 km'lik bir kare

\begin{minted}[fontsize=\footnotesize]{python}
u_wind = np.load('uwind.npz')['arr_0']
v_wind = np.load('vwind.npz')['arr_0']
gi,gj = u_wind.shape
cell_count = gi*gj
area = 2000*1e9 # bu alani veri alirken tanimlamis olduk
cell_area = area / cell_count

wspeedsquare = u_wind**2+v_wind**2
wspeedsquare = wspeedsquare.reshape(-1)
wspeedsquare = wspeedsquare[wspeedsquare > 30.0]
IKE = np.sum(0.5*wspeedsquare*cell_area) / 1e12
print (np.round(IKE,2), 'terrajoule')
\end{minted}

\begin{verbatim}
340.98 terrajoule
\end{verbatim}



Kaynaklar

[1] {\em Wired}, \url{https://www.wired.com/2012/11/what-is-the-true-measure-of-a-storm}

\end{document}
