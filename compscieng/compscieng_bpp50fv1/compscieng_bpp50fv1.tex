\documentclass[12pt,fleqn]{article}\usepackage{../../common}
\begin{document}
Sınırlı Hacim (Finite Volume) Yöntemi - 1

Üç boyutlu kütle muhafazası üzerinden süreklilik formül [2]'de işlendi.  Şimdi
tek boyutlu ortamda muhafaza kanunlarını işleyeceğiz, gaz dinamiği, genel
aerodinamik konularında bu yaklaşım faydalı olacak. Çözmeye çalışılacak
problemler muhafaza kanunları içeren hiperbolik sistemler (hyperbolic systems of
conservation laws) olarak anılır. Bu tür sistemler zamana bağlı çoğunlukla gayrı
lineer kısmı türevsel denklemlerdir, ve aslında basit yapıları vardır. Tek
yersel boyutta bu denklemler şuna benzer [3, sf. 1],

$$
\frac{\partial }{\partial t} u(x,t) + 
\frac{\partial }{\partial x} f(u(x,t)) = 0
\mlabel{1}
$$

Daha önce [1]'de Burgers'in denklemini görmüştük, 

$$
u_t + uu_x = 0
\mlabel{2}
$$

Bu denklem (1) ışığında düşünülebilir, eğer $f(u) = \frac{1}{2}u^2$ tanımlarsak,
(1) formülü, yani $u_t + f(u)_x = 0$, formül (2) ile aynıdır.

Üstteki türden denklemleri tek boyutta çözmeyi işleyeceğiz öncelikle, çünkü çok
boyutta çözüm tek boyuta indirgenerak yapılabiliyor.

Hiperbolik denklemleri çözmek için entegral formu işlemek lazım, çünkü aslında
diferansiyel form entegral formden türetilmiş ve bu türetim pürüzsüzlük
faraziyesi üzerinden yapılmış. Bu sebeple mesela özellikle kesintili başlangıç
şartları var ise diferansiyel formun her yerde düzgün işlemesi mümkün değil,
çünkü kesintilerde türev alınamıyor. O zaman once entegral formu gorelim.

İçinde gaz olan sadece tek boyutuna baktığımız bir tüp düşünelim, $x$ tüpün
üzerindeki bir noktayı temsil edecek, $\rho(x,t)$ ise tüpün $x$ noktasında ve
$t$ anındaki yoğunluğunu verecek diyelim. Yoğunluğu kullanarak $x_1$ ve $x_2$
noktaları arasındaki $t$ anındaki kütle

$$
\int _{x_1}^{x_2} \rho(x,t) \ud x
$$

ile hesaplanabilir. Tüpün duvarları tam izole ise ve kütle yoktan varedilip
yokedilemeyeceğine göre tupe gaz giriş ya da çıkış sadece $x_1,x_2$
noktalarından olabilir [3, sf. 14]. Şimdi bir gaz hareket hızı düşünelim,
$v(x,t)$ ile, o zaman gaz akma oranı, ya da akış (flux)

$$
flux = \rho(x,t) v(x,t)
$$

olur. Üstteki fiziksel kurallardan hareketle $[x_1,x_2]$ deki kütlenin
değişim oranı $x_1$ ve $x_2$ noktalarındaki akışın farkına eşit olmalıdır,

$$
\frac{\ud}{\ud t} \int _{x_1}^{x_2} \rho(x,t) \ud x =
\rho(x_1,t) v(x_1,t) - \rho(x_2,t) v(x_2,t)
$$

İşte bu muhafaza kanununun entegral formudur. 

Ustteki formulu $t_1,t_2$ zaman araligi icin entegre edersek, ki boylece
bu zaman icindeki tum toplam akisi hesaplayabilelim, o zaman

$$
\int_{t_1}^{t_2} \left( \frac{\ud}{\ud t} \int _{x_1}^{x_2} \rho(x,t) \ud x  \right)  =
\int_{t_1}^{t_2} \rho(x_1,t) v(x_1,t) - \int_{t_1}^{t_2}  \rho(x_2,t) v(x_2,t) 
$$

Soldaki kısım zaman üzerinden türevin yine zaman üzerinden entegrali, o zaman
yokolabilir, Calculus'un Temel Teorisi üzerinden basitleştirirsek,

$$
\int _{x_1}^{x_2} \rho(x,t_2) \ud x - \int _{x_1}^{x_2} \rho(x,t_1) \ud x  = 
\int_{t_1}^{t_2} \rho(x_1,t) v(x_1,t) - \int_{t_1}^{t_2}  \rho(x_2,t) v(x_2,t) 
$$

Ufak bir yer değiştirme sonrası

$$
\int _{x_1}^{x_2} \rho(x,t_2) \ud x =
\int _{x_1}^{x_2} \rho(x,t_1) \ud x  +
\int_{t_1}^{t_2} \rho(x_1,t) v(x_1,t) -
\int_{t_1}^{t_2}  \rho(x_2,t) v(x_2,t) 
$$

Üstteki form ileride tekrar lazım olacak, zaman adımı atmaya uğraştığımız
hesapsal yöntemlerde $t_1$ ve $t_2$ üzerinden bir entegral, hesabı bir sonraki
zamana geçirmeye uğraştığımızda, adım attığımızda yardımcı oluyor.

Neyse şimdi diferansiyel forma geçise dönelim. Bu noktada $\rho(x,t)$ ve
$v(x,t)$'nin türevi alınabilir fonksiyonlar olduğunu farz ediyoruz. Şimdi
üstekini yine ufak bir değişim sonrası

$$
\int _{x_1}^{x_2} \rho(x,t_1) \ud x  +
\int _{x_1}^{x_2} \rho(x,t_2) \ud x -
\int_{t_1}^{t_2}  \rho(x_2,t) v(x_2,t) -
\int_{t_1}^{t_2} \rho(x_1,t) v(x_1,t) = 0
\mlabel{3}
$$

olarak görelim. Eğer Calculus'un Temel Teorisi ile ilk iki terime
$\int_{t_1}^{t_2} .. \ud / \ud t$ son iki terime $\int_{x_1}^{x_2} .. \ud / \ud x$
ekleyebilirsek, tüm terimlerde aynı entegraller olacağı için, 
$\int_{t_1}^{t_2} \int_{x_1}^{x_2} $ altında tüm terimleri gruplayıp
basitleştirmek mümkün, ve bunlar sıfıra eşit olur. Bu bizi diferansiyel
forma götürebilir. Yani

$$
\rho(x,t_2) - \rho(x,t_1) = \int_{t_1}^{t_2}
\frac{\partial }{\partial t} \rho(x,t) \ud t
$$

ve

$$
\rho(x_2,t)v(x_2,t) - \rho(x_1,t)v(x_1,t) =
\int _{x_1}^{x_2} \frac{\partial }{\partial x} (\rho(x,t)v(x,t)) \ud x
$$

eşitliklerinden hareketle, bunları (3)'e uygulayıp

$$
\int _{t_1}^{t_2} \int _{x_1}^{x_2}  \left\{
\frac{\partial }{\partial t} \rho(x,t)  +
\frac{\partial }{\partial x} (\rho(x,t)v(x,t))
\right\} \ud x \ud t = 0
$$

elde ediyoruz. Bu ifadenin $[x_1,x_2]$ ve $[t_1,t_2]$ arasındaki tüm değerlerde
doğru olması gerektiği için entegre edilenin sıfır olması gerekiyor ([2]'dekine
benzer bir mantık yürütüldü), yani

$$
\rho_t + (\rho v)_x = 0
$$

olmali. Boylece kutlenin muhafaza kuralini diferansiyel formda elde etmis olduk.

Bu form izole formda çözmenin tek yolu $v$'nin önceden bilindiği durumdadır, ya
da $v$ fonksiyon $\rho(x,t)$'ye bağlı bir fonksiyon olmalıdır, yani
$f(\rho) = \rho v$ gibi. Bu durumda üstteki ifade $\rho$ için tek sayısal
muhafaza kanunu haline gelir,

$$
rho_t + f(\rho)_x = 0
$$











[devam edecek]

Kaynaklar

[1] Bayramlı, {\em Hesapsal Bilim, Hesapsal Sıvı Dinamiğine Giriş}

[2] Bayramlı, {\em Fizik, Sıvılar, 1}

[3] Leveque, {\em Numerical Methods for Conservation Laws}

\end{document}
