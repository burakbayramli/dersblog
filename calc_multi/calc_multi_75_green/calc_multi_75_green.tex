\documentclass[12pt,fleqn]{article}\usepackage{../../common}
\begin{document}
Green'in Teorisi, Uzaklaşım, Stokes










Gauss-Green Eşitliği

Gauss-Green eşitliği iki boyutta şu şekilde gösterilebilir [1, sf. 262],

$$
\iint_R (\nabla u ) \cdot w \ud x \ud y =
\iint_R u (- \bdiv w) \ud x \ud y + \int_C u w \cdot n \ud s
$$

Türetmek için başlangıç noktası $uv$ üzerinde uzaklaşım almak. Aslında
ileride göreceğimiz gibi çok boyutta parçalı entegral tekniği Gauss-Green'in
uzantısı bir bakıma ve tek boyutta gördük ki [3] parçalı entegrale erişmek
için de Calculus'un çarpım kuralından başlanmıştı.

$$
\bdiv (uw) = \bdiv (u w_1 + u w_2) =
\frac{\partial u}{\partial x} w_1 +
\frac{\partial w_1}{\partial x} u +
\frac{\partial u}{\partial y} w_2 +
\frac{\partial w_2}{\partial y} u 
$$

Gruplarsak,

$$
= \left( 
\frac{\partial u}{\partial x} w_1 +
\frac{\partial u}{\partial y} w_2 \right) +
\left( 
\frac{\partial w_1}{\partial x} u +
\frac{\partial w_2}{\partial y} u \right)
$$

Daha kısa şekilde,

$$
\bdiv (uw) = \nabla u \cdot w + u \bdiv(w)
$$

Üstteki ifade üzerinde Uzaklaşım Teorisi'ni uygulayalım. Önce
$\iint_R \bdiv (uw)$,

$$
\iint_R \bdiv (uw) \ud x \ud y= \iint_R \nabla u \cdot w + u \bdiv(w) \ud x \ud y
$$

$$
= \iint_R \nabla u \cdot w  \ud x \ud y + \iint_R u \bdiv(w) \ud x \ud y
$$

Uzaklaşım Teorisi'ne göre sağ taraf $\int_C uw \cdot n \ud s$ olmalı, yani

$$
\iint_R \nabla u \cdot w  \ud x \ud y + \iint_R u \bdiv(w) \ud x \ud y = \int_C uw \cdot n \ud s
$$

Eşitliğin sol tarafındaki ikinci terimi sağa geçirirsek,

$$
\iint_R \nabla u \cdot w  \ud x \ud y =
\iint_R u (-\bdiv w) \ud x \ud y + \int_C uw \cdot n \ud s
$$

[1] notasyonu ile $\nabla$ yerine $\grad$,

$$
\iint_R \grad u \cdot w  \ud x \ud y =
\iint_R u (-\bdiv w) \ud x \ud y + \int_C uw \cdot n \ud s
\mlabel{3}
$$

Böylece Gauss-Green eşitliğine erişmiş olduk.

Green'in İlk Eşitliği 

Eğer (3) içinde $w$ için $\grad u$ sokarsak, bu bize Green'in İlk Eşitliği (Green's First
İdentity) denen formülü veriyor [1, sf. 281], 

$$
\iint_R \grad u \cdot \grad u  \ud x \ud y =
\iint_R u (-\bdiv \grad u) \ud x \ud y + \int_C u \grad u \cdot n \ud s
$$

Gradyanın uzaklaşımı bazen $\Delta$ notasyonu ile gösterilir, öyle yapalım,

$$
\iint_R | \grad u |^2  \ud x \ud y = - \iint_R u (\Delta u) \ud x \ud y +
\int_C u \grad u \cdot n \ud s
$$

Eşitliğin sağından, solundan birkaç yer değişim sonrası,

$$
\iint_R u (\Delta u) \ud x \ud y =
- \iint_R | \grad u |^2  \ud x \ud y
+ \int_C u \grad u \cdot n \ud s
$$

Böylece [1, sf. 281]'daki forma erişmiş olduk. Bu Green'in İlk Eşitliği.

[devam edecek]

Kaynaklar

[1] Strang, {\em Computational Science and Engineering}

[2] Bayramli, {\em Cok Degiskenli Calculus, Ders 23}

[3] Bayramlı, {\em Diferansiyel Denklemler, Ekler}

\end{document}
