\documentclass[12pt,fleqn]{article}\usepackage{../../common}
\begin{document}
Ders 2.9

Formüllerimizin tekrar üzerinden geçelim,

$$
\frac{\partial u}{\partial t} +
\frac{\partial }{\partial x} f(u) = 0
\mlabel{1}
$$

$$
\frac{\ud}{\ud t} \int_{x_L}^{x_R} u \ud x + f(u_R) - f(u_L) = 0
\mlabel{2}
$$

$$
u(x,t) = u(x - f'(u) t, 0)
\mlabel{3}
$$

İki formül tek sayısal, tek boyutta muafaza kanununu gösteriyor. Yer tek
boyutlu, sadece $x$ var. Akış fonksiyonu $f(u)$. (1)'de diferansiyel form var ve
gördük ki bu form bir süreksiz (discontinuous) çözüm, süreksizlik ortaya
çıkarıyor. Bu sebeple denklemin entegre edilmiş formu (2) ile iş yapmaya karar
verdik. Mesela $u$'da yoğunluk olsaydı entegrasyon bize iki nokta arasındaki
kütleyi verirdi (çünkü kütle, yoğunluğun bir hacim üzerinden toplanmış halidir),
o zaman (2)'deki entegralin türevi kütlenin zamana göre değişimi olurdu, ve
diğer iki terimi de göz önüne alırsak, bu değişim sadece dışarı çıkan, eksi
giren akışa eşit olurdu. Muhafaza kanunu budur. 

Diferansiyel denklemin çözümü (3) ile karakteristik çizgiler ortaya çıktı,
onları başlangıç zamanı $t=0$'dan takip edebiliyorduk. Başlangıçtaki değerleri
çizgi boyunca taşınıyordu, problem karakteristikler çakıştığında ya da aynı
başlangıçtan farklı yönlere yayıldığında / dağıldığında (fan out) ortaya
çıkıyordu. Bu iki farklı durumu altta görüyoruz.

\includegraphics[width=20em]{compscieng_2_09_01.png}

İki farklı problem üsttekiler ve ikisinin de farklı tedavisi var. Soldaki trafik
örneğinde kırmızı ışık yandığındaki durum, ışığa gelen arabalar orada tıkanıyor,
ışığın öteki tarafında yoğunluk az, ama gerisinde durum farklı, ve arabalar
hızlı bir şekilde durmalılar. O soru işareti olan yeri düşünelim, aynı noktaya
farklı yönlerden gelen iki değer nereye gider? Bu durumu halledecek bir kural
lazım.

Aynı şekilde üst sağdaki durum için bir kural lazım. Orada soru işareti
boşlukta, ama orası bir değeri temsil ediyor, oraya nasıl gelinir, hiç bir
karakteristik oraya gitmiyorsa?

















[devam edecek]

\end{document}
