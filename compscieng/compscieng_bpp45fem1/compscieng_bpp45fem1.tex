\documentclass[12pt,fleqn]{article}\usepackage{../../common}
\begin{document}
Sonlu Öğeler Metotu (Finite Elements Method -FEM-) - 1

Galerkin, Ağırlıklı Artıklar Metotu (Weighted Residual Method)

WRS diferansiyel denklemleri yaklaşık olarak çözebilen bir metottur, bunu tüm
yaklaşık fonksiyonun tanım kümesi üzerinden ağırlıklı hata artıklarını sıfıra
eşitleyerek yapar. Bununla ne demek istediğimizi birazdan yakından göreceğiz
[1, Ders 3].

Mesela [2]'de öğrendiğimiz Euler-Bernoulli kirişlerini tanımlayan denklemi
hatırlarsak,

$$
E I \frac{\ud^4 y}{\ud X_1^4} = q
$$



















[1] Petitt, {\em Intro to the Finite Element Method}, University of Alberta,
    \url{https://www.youtube.com/watch?v=2iUnfPRk6Ro&list=PLLSzlda_AXa3yQEJAb5JcmsVDy9i9K_fi}

[2] Bayramlı, {\em Fizik, Materyel Mekanigi - 2}
    
\end{document}





