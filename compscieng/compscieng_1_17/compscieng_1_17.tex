\documentclass[12pt,fleqn]{article}\usepackage{../../common}
\begin{document}
Ders 1.17

[Giriş Konuları Atlandı]

Sonlu Öğeler Metodu (Finite Elements Method)

FEM yontemiyle diferansiyel denklem cozerken once denklemin zayif formu elde
edilir. Dikkat, bu formu daha sonlu ogeleri ortaya atmadan, matrislerden vs
bahsetmeden yazmak gerekir. Baslangic diferansiyel denklem ve onun belli
sartlarda esit oldugu (ama cozum icin faydali olabilecek) baska bir formu
ortaya cikartmaktan bahsediyorum. 

Zayif formu ``kuvvetli formdan'' cikartiyoruz, kuvvetli form diferansiyel
denkleminin ilk hali,

$$
- \frac{\ud}{\ud x} \left( c(x) \frac{\ud u}{\ud x} \right) = f(x)
$$

Zayif forma gecmek icin esitligin iki tarafini ona $v$ sembolu verecegim bir
``test fonksiyonu'' ile carpiyorum. Dikkat, $u$ cozum, $v$ ustteki formulu
``test'' etmek icin kullandigim herhangi bir fonksiyon. 

$$
- \frac{\ud}{\ud x} \left( c(x) \frac{\ud u}{\ud x} \right) v(x) =
f(x) v(x)
$$

Sonra usttekini entegre ediyorum,

$$
\int_{0}^{1} - \frac{\ud}{\ud x} \left( c(x) \frac{\ud u}{\ud x} \right) v(x) \ud x=
\int_{0}^{1} f(x) v(x) \ud x
$$

Boylece zayif forma erismis olduk. Not: herhangi bir $v(x)$ dedik ama orada
bazi sartlar olabilir, ileride gorecegiz. 

Ustteki formul her $v(x)$ icin dogru olmaliysa belki $v(x)$'in bir alanda
konsantre olmasini zorlayabilirdim, sonra baska bir $v(x)$ denerdim belki onun
baska noktalara konsante olmasini saglayabilirdim... Usttekinden neler
cikartabilirim buradan diye dusunuyorum su anda.. Fakat sundan eminim ki sol
taraf sag tarafa esit olmali. Bunu kullanarak bir suru takla attiktan sonra bile
guclu forma donebilecegimi biliyorum.

Ama biz su anda zayif formu seviyoruz.. onu sevmeyi ogrenmemiz lazim, bize
pek cok bazi ek yetenekler saglayacak cunku. Esitligin sol tarafinda neler
yapabilirim mesela? Sag taraf tamam, orasi kafama uygun. 







[devam edecek]

\end{document}
