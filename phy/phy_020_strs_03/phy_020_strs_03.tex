\documentclass[12pt,fleqn]{article}\usepackage{../../common}
\begin{document}
Materyel Mekaniği - 3

İlerlemeden önce birazdan lazım olacak iki formülü bulalım. Daha önce gördük ki

$$
I = \int \int X_2^2 \ud X_2 \ud X_3
$$

ve

$$
M = \int \int -X_2 \sigma_{11} \ud X_2 \ud X_3
$$

Üstteki iki formülü birleştirelim, $M$ içinde $I$ oluşturalım, ve yerine koyalım,

$$
M = \int \int -\frac{X_2}{X_2^2} X_2^2 \sigma_{11} \ud X_2 \ud X_3 =
-\frac{1}{X_2} \sigma_{11} I 
$$

Buradan $\sigma_{11}$ eşitliğine geçilir,

$$
\implies \sigma_{11} = -\frac{M X_2}{I}
$$

Yatay Kaykılma Stresi

Kaykılma stresi $\tau$'yu bulmak için yine çubuğun ufak bir kısmına odaklanalım,

\includegraphics[width=15em]{phy_020_strs_00_06.jpg}

Tüm etki eden kuvvetlerin toplamı sıfır olmak zorundadır [2],

$$
-P + (P + \ud P) + \tau b \ud x = 0
$$

$$
-\ud P/\ud x = \tau b
\mlabel{1}
$$

\includegraphics[width=15em]{phy_020_strs_00_07.jpg}

$P$'yi bulmak için $A$ bölgesindeki stresleri entegre ediyoruz,

$$
\int_A \ud P = \int_A \sigma_b \ud A
$$

Fakat daha önce bulduk ki $\sigma_b = -My / I$, yerine koyunca,

$$
P = \int_A - \frac{My}{I} \ud A
$$

$M$ ve $I$ sabittir, entegral dışına çıkartılabilir,

$$
P = - \frac{M}{I} \int_A y \ud A = - \frac{MQ}{I}
$$

Üstte bulunan $P$'yi (4)'e sokunca,

$$
- \frac{\ud}{\ud x} \left( - \frac{MQ}{I} \right) = \tau b
$$

$$
\frac{Q}{I} \frac{\ud M}{\ud x} = \tau b
$$

Şimdi hatırlarsak $\ud M/\ud x$ türevi yatay kaykılma yükü $V$'ye eşittir. O
zaman

$$
\frac{Q}{I} V = \tau b
$$

Nihai yatay kaykılma stres denklemi,

$$
\tau = \frac{V Q}{I b}
$$
























Euler-Bernoulli çubuklarını temel alan analizleri üç adıma bölmek mümkündür.

1) Uygulanan yük $q$'yu kullanarak saptırma (deflection) fonksiyonu $y$'yi hesapla,

$$
q = E I \frac{\ud^4 y}{\ud X_1^4}
$$

Formülde görüyoruz eğer $q$ biliniyorsa ve elde yeterli sınır şartları var ise
(dört tane) diferansiyel denklemi kullanarak $y$'yi bulabiliriz
[1, Lecture 2, 2:02:00]. 

2) Saptırma $y$ bulunduktan sonra onu kullanarak kaykılma (shear) ve bükülme
momentini hesapla,

$$
M = E I \frac{\ud^2 y}{\ud X_1^2}, \quad
V = E I \frac{\ud^3 y}{\ud X_1^3} 
$$

çünkü sonuçta $M,V$ hesapları $y$'nin birer fonksiyonu, elde edilen $M,V$
sonuçları $X_1$'in fonksiyonları olacak tabii ki.






[devam edecek]

Kaynaklar

[1] Petitt, {\em Intro to the Finite Element Method}, University of Alberta,
    \url{https://www.youtube.com/watch?v=2iUnfPRk6Ro&list=PLLSzlda_AXa3yQEJAb5JcmsVDy9i9K_fi}

[2] Gramoll, {\em Mechanics},
    \url{http://www.ecourses.ou.edu/cgi-bin/ebook.cgi?topic=me}

\end{document}
