\documentclass[12pt,fleqn]{article}\usepackage{../../common}
\begin{document}
Materyel Mekaniği - 2

Sonsuz Küçük (Infinitesimal) Gerilme Tensörü

Green gerilme tensörünü gördük, kuvvetli bir yaklaşım ama bizim daha çok
kullanacağımız şimdi anlatacağımız. Niye? Çünkü Green tensörü sonlu değişimler
için geçerli ama çoğu uygulamada bize lazım olan çok ufak yamulmalar. Ufak
değişimler derken, önceki dersteki (3) formülünden hareketle, oradaki en son
terimi hatırlarsak, çok ufak yamulmalar için $\nabla u^T \nabla u << \nabla u$
olur, yani ufak değişimlerde o karesel işlem $\nabla u$'dan daha ufak sonuç
verir. O zaman belli durumlarda son terim yaklaşık sıfır kabul edilebilir,
$\nabla u^T \nabla u \approx 0$. O zaman Green tensörü bu durumlarda yaklaşık
olarak alttaki gibi olur,

$$
\epsilon_{Green} \approx \frac{1}{2} (\nabla u + \nabla u^T )
$$

Bileşen formunda

$$
\epsilon_{ij} = \left(
\frac{\partial u_i}{\partial X_j} + \frac{\partial u_j}{\partial X_i}
\right)
$$

Bu tensör de simetrik, fakat sadece ufak şekil değişimleri, yamulmalar için
geçerli. Fakat zaten, mesela inşaat mühendisliği durumunda, binalar, demir
çubuklar (beam) ile iş yaptığımız zaman, bu tür şekil değişimi faraziyesi
yeterli. Çünkü eh, biraz düşünürsek eğer binamız büyük şekil değişimleri
yaşıyorsa önümüzde daha büyük bir problem var demektir.

Gerilme tensörü aslında bu konunun en zor bölümü denebilir; eğer öğrenciler bunu
anlarsa, konunun geri kalanı kolay olacak artık.

Cauchy Stres Tensörü

Gerilme tensöründen stres tensörlerine geldik. İlk önce çekiş (traction) ya da
stres vektöründen bahsedeceğiz. Diyelim ki elimizde bir çubuk var, onu ortadan
kestiğimizi düşünelim, ve iki parça ortaya çıkıyor. Şimdi belki lisans seviyesi
Statik dersinden hatırlayanlar olabilir, bir nesneyi (sanal olarak) kesince onun
iç kuvvetlerini serbest bırakmış oluyoruz. 

\includegraphics[width=20em]{phy_020_strs_02_01.png}

Kesit düzleminden bahsedelim önce, kesit tam dik olabilir ama bu şart değil,
nasıl olursa olsun o düzleme dik olan bir $\vec{n}$ vektörü ile bu kesitin
duruşunu temsil edebiliriz. 

Bu serbest bırakılmış iç kuvvetler darmadağın gözüküyor. Bir $\Delta A$
alanı tanımlayıp o alandaki tüm kuvvetleri alıp toplarsak bir $\Delta F$
elde edebiliriz, bu tek vektör daha derli toplu.

\includegraphics[width=20em]{phy_020_strs_02_02.png}

$\vec{n}$ ile tanımlı bir nokta etrafındaki düzlemin çekiş vektörünü (yani bir
noktadaki stres vektörü) şimdi şöyle tanımlıyoruz,

$$
t_n = \lim_{\Delta A \to 0} \frac{\Delta F}{\Delta A}
$$

Newton'un hareket kanunu üzerinden tabii ki sol taraftaki çekiş ile sağ
taraftaki birbirini dengelemeli, $t_n = -t_{-n}$.

Çekiş vektörü için formel tanım böyle. Ama kimse formel tanımı pek sevmiyor
sözel şekilde anlatırsak, çubuğu aldım ve kestim, Statik dersi der ki kaykılma
(shear), normal kuvvet ve eğilme momentimi böylece elde ederim. Bu üç boyutlu
nesnelerde olan şudur, çubuğu kesiyorum ve bileşenleri stres öğeleri olan tek
bir vektör elde ediyorum.

Şimdi çekiş vektörü kavramını daha basitleştirmeye uğraşalım. Bunun için
patatesimize geri dönüyoruz. Patatesten üç boyutlu sonsuz küçük küp şeklinde bir
parça çıkarttığımızı düşünelim şimdi,

\includegraphics[width=15em]{phy_020_strs_02_03.png}

Bu ufak parça nesnenin bütünlüğünden çıkartıldığı için çekiş vektörlerinin
bu küpün yüzlerine etki eden stres vektörleri olduğunu söyleyebiliriz.

Küp sekli iyi bir seçim aslında çünkü her yüz kordinat eksendeki bir baz düzleme
paralel. Ayrıca $t_{e_1}$, $t_{e_2}$, $t_{e_3}$ yerine de daha iyi bir temsil
şekli bulabiliriz, küpün her yüzündeki bu $t$ çekiş vektörlerini de üç parçaya
ayırabiliriz,

\includegraphics[width=10em]{phy_020_strs_02_04.png}

Bu şekildeki temsilin iyi bir tarafı her yüzdeki üç vektörün orijindeki
baz vektörlerle birebir uyuşması. O zaman mesela $t_{e_3}$'u o baz vektörlerin
lineer bir kombinasyonu olarak yazabilirim,

$$
t_{e_3} = \sigma_{31} e_1 + \sigma_{32} e_2 + \sigma_{33} e_3
$$

Üsttekini herhangi bir yüzey için yazarsak, yani $e_i$'in dik olduğu bir yüzey
için

$$
t_{e_i} = \sigma_{ij} e_j
$$

Einstein notasyonu kullandık, bu notasyonla her $i$ için mümkün tüm $j$'lerin üç
tane terimi ortaya çıkardığı kabul edilir.

Küpün yüzlerindeki çekiş vektörünü gösterebiliyoruz, fakat acaba herhangi bir
yöne bakan bir yüzey için stres vektörü ne olurdu? Bu ifadeyi genel bir şekilde
yazmak mümkün, hem bunu göstermek (ve ileride ispat etmek) için Cauchy Stres
Tetrahedon'u denen bir kurguyu anlatmamız lazım.

\includegraphics[width=20em]{phy_020_strs_02_05.png}

Tetrahedon üstteki gibi tanıdık bir şekil. Cauchy Lemma'sı ve Cauchy Kanununa
göre

$$
t_n = \sigma^T n
$$

olarak belirtilebilir, detaylı olarak belirtirsek,

$$
\left[\begin{array}{ccc}
t_{n1} \\ t_{n2} \\ t_{n3} 
\end{array}\right] =
\left[\begin{array}{ccc}
\sigma_{11} & \sigma_{21} & \sigma_{31} \\
\sigma_{12} & \sigma_{22} & \sigma_{32} \\
\sigma_{13} & \sigma_{23} & \sigma_{33} 
\end{array}\right]
\left[\begin{array}{ccc}
n_1 \\ n_2 \\ n_3
\end{array}\right]
$$

Bu formüldeki $\sigma$ Cauchy stres tensörü olarak adlandırılır. Üstteki ifade
şunu söylemiş oluyor aslında, üstünde kuvvet etkileri olan bir katı cisim bize
verilince yönden bağımsız bir Cauchy stres tensörü $\sigma$ elde edebiliriz,
yani öyle bir $\sigma$ vardır ki $n$ yönündeki $t_n$ elde etmek için
$t_n = \sigma^T n$ yapılabilir.

Cauchy tensörü hakkında bazı ispatlar için [1, sf. 330]. 

Cauchy tensörünün bazı özellikleri,

1) $\sigma$ simetriktir, yani $\sigma_{ij} = \sigma_{ji}$.

2) Öyle bir kordinat sistemi vardır ki bu sistemde $\sigma$ köşegendir. Lineer
cebirde köşegenleştirme vardır bildiğimiz gibi, burada o teknik uygulanır, bu
şekilde ana bileşen stresleri (principle stresses) denen stres vektörleri elde
edilebilir.

3) Feragat / teslim / esneme yüzeyi (yield surfaces) denen bir hesabı bu tensör
üzerinden yapmak mümkün. 

Normal ve Kaykılma Stresleri

Çekiş vektörü kavramının bir daha üzerinden geçelim, formülü gördük ama
kavramsal olarak neyden bahsediyoruz? İki boyuttaki kütleye bakalım, onu
herhangi bir düzlemle kestim diyelim, 

\includegraphics[width=15em]{phy_020_strs_02_06.png}

Kesim yüzeyini normal vektörü ile tanımlıyoruz. Çekiş vektörü bir objenin iç
streslerinin bir yüzeye yansımasıdır. Dikkat, bu iç streslerin, kuvvetlerin
vektörsel bileşimleri illa ki yüzeyin normal vektörü ile aynı yönde olmayabilir,
onun için üstteki şekilde farklı yöne bakan bir $t_n$ gösteriliyor. Muhakkak
diğer parçada ona eşit dengeliyici reaksiyon var, vs.

Simdi cekis vektorunun bilesenlerine gelelim, $t_n$ daha once belirttik stres
tensoru carpi $n$. 

\includegraphics[width=10em]{phy_020_strs_02_07.png}

Bilesen baglaminda $t_n$'i alip $n$ uzerine skalar olarak yansitabilirim,
$\sigma_n$ buyuklugunu elde ederim, ona normal stres diyelim. Bu bir skalar
buyukluk, hesabi

$$
\sigma_n = \proj_n t_n = \frac{t_n \cdot n}{||n||}
$$

Üstteki hesap aslında daha da basitleşebilir çünkü $n$'nin norm'ü 1'e
eşittir yani $||n||=1$, o zaman geriye sadece $\sigma_n = t_n \cdot n$
kalıyor.

Yüzeye paralel olan bileşen kaykılma stresi (resimde kırmızı renkli $\tau_n$)
için artık basit bir vektör çıkartım işlemi yeterli, resme göre

$$
t_n = \vec{\sigma}_n + \vec{\tau}_n 
$$

olduğuna göre basit bir tekrar düzenleme sonrası,

$$
\vec{\tau}_n  = \vec{t}_n - \vec{\sigma}_n 
$$

elde edilir. Yanliz $\vec{\sigma}_n$ vektoru kullanildi, daha once $\sigma_n$
skalar demistik, vektoru nasil elde ediyoruz? Bu basit, $\vec{\sigma}_n$
yonsel olarak normal vektor $n$, ya da $\vec{n}$ ile ayni yonde olacagi
icin, eh $\vec{n}$ zaten birim vektordur o zaman $\vec{\sigma}_n$ icin
$\vec{n}$ ile $\sigma_n$ carpilmasi yeterli,

$$
\vec{\tau}_n  = \vec{t}_n - \sigma_n n
$$











[devam edecek]

Kaynaklar

[1] Kelly, Solid Mechanics Part III, Auckland University

\end{document}
