\documentclass[12pt,fleqn]{article}\usepackage{../../common}
\begin{document}
Materyel Mekaniği - 2

Sonsuz Küçük (Infinitesimal) Gerginlik Tensörü

Green gerginlik tensorunu gördük, kuvvetli bir yaklaşım ama bizim daha çok
kullanacağımız şimdi anlatacağımız. Niye? Çünkü Green tensoru sonlu değişimler
için geçerli ama çoğu uygulamada bize lazım olan çok ufak yamulmalar. Ufak
değişimler derken, (3) formülünden hareketle, oradaki en son terimi hatırlarsak,
çok ufak yamulmalar için $\nabla u^T \nabla u << \nabla u$ olur, yani ufak
değişimlerde o karesel işlem $\nabla u$'dan daha ufak sonuç verir. O zaman belli
durumlarda son terim yaklaşık sıfır kabul edilebilir, $\nabla u^T \nabla u
\approx 0$. O zaman Green tensörü bu durumlarda yaklaşık olarak alttaki gibi
olur,

$$
\epsilon_{Green} \approx \frac{1}{2} (\nabla u + \nabla u^T )
$$

Bilesen formunda

$$
\epsilon_{ij} = \left(
\frac{\partial u_i}{\partial X_j} + \frac{\partial u_j}{\partial X_i}
\right)
$$

Bu tensor de simetrik, fakat sadece ufak şekil değişimleri, yamulmalar için
geçerli. Fakat zaten, mesela inşaat mühendisliği durumunda, binalar, demir
çubuklar (beam) ile iş yaptığımız zaman, bu tür şekil değişimi faraziyesi
yeterli. Çünkü, eh, biraz düşünürsek eğer binamiz büyük şekil değişimleri
yaşıyorsa önümüzde daha büyük bir problem var demektir. 

Gerginlik tensoru aslında bu konunun en zor bölümü denebilir; eğer öğrenciler
bunu anlarsa, konunun geri kalanı kolay olacak artık.





[devam edecek]

\end{document}
