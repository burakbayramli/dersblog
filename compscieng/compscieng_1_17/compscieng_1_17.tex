\documentclass[12pt,fleqn]{article}\usepackage{../../common}
\begin{document}
Ders 1.17

[Giris Konulari Atlandi]

Sonlu Öğeler Metodu (Finite Elements Method)

FEM yontemiyle diferansiyel denklem cozerken once denklemin zayif formu elde
edilir. Dikkat, bu formu daha sonlu ogeleri ortaya atmadan, matrislerden vs
bahsetmeden yazmak gerekir. Baslangic diferansiyel denklem ve onun belli
sartlarda esit oldugu (ama cozum icin faydali olabilecek) baska bir formu
ortaya cikartmaktan bahsediyorum. 




[devam edecek]

\end{document}
