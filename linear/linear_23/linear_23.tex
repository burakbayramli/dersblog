\documentclass[12pt,fleqn]{article}\usepackage{../../common}
\begin{document}
Ders 23

Bu derste birinci dereceden, sabit katsayılı (coefficients) lineer
diferansiyel denklem sistemini çözmeye göreceğiz. Form

$$ \frac{du}{dt} = A u $$

şeklinde olacak. Katsayılar $A$ matrisi içinde. Temel fikir: sabit katsayılı
lineer diferansiyel denklemlerin çözümü üsteldir (exponential). Sonucun
üstel olduğunu bilince bulmamız gereken üstel değerin ne olduğu, yani $e$
üzerine ne geldiği, onu neyin çarptığı, ki bunları bulmak lineer cebirin
işi olacak.

Bir örnekle başlayalım

$$ \frac{du_1}{dt} = -u_1 + 2u_2 $$

$$ \frac{du_2}{dt} = u_1 - 2u_2 $$

Üstteki sistemin katsayılarını dışarı çekersek $A$ matrisi şu olur

$$ A = 
\left[\begin{array}{cc}
-1 & 2 \\
1 & -2
\end{array}\right]
 $$

Başlangıç değerleri şöyle olsun

$$ u(0) = 
\left[\begin{array}{c}
1 \\
0
\end{array}\right]
 $$

Eğer $u_1$ ve $u_2$'yi bu sistemin temsil ettiği iki kap gibi görsek
(mesela), her şey $u_1$'in ``içinde'' olarak başlayacaktı, sonra zaman
geçtikçe oradan çıkacak, $u_2$'ye doğru akacak. Tüm bunları $A$ matrisinin
özdeğer/vektörlerine bakarak anlayabilmemiz lazım. O zaman ilk işimiz
özdeğer/vektörleri bulmak olmalı. 

$A$'ya bakınca ne görüyoruz? Bir kolon diğerinin katı, o zaman matris
eşsiz (singular), bu demektir ki bir özdeğer $\lambda = 0$. Diğer
özdeğer için bir numara kullanalım; özdeğerlerin toplamı matris izine
(trace) yani çaprazdaki sayıların toplamına eşit olduğuna göre, ve toplam
$=-3$, o zaman ikinci özdeğer $-3$ olmalı. Özvektörler ise

$$ 
\left[\begin{array}{c}
2 \\ 1
\end{array}\right] 
,
\left[\begin{array}{c}
1 \\ -1
\end{array}\right]
 $$

O zaman çözüm iki çözümün toplamı olacak

$$ u(t) = c_1 e^{\lambda_1 t}x_1 + c_2 e^{\lambda_2 t}x_2$$

Bu genel çözüm. Genel çözüm $c_1, c_2$ haricindeki birinci ve ikinci terimdeki
``pür üstel'' çözümlerden oluşuyor. Hakikaten mesela $ e^{\lambda_1 t}x_1 $
diferansiyel denklemi çözüyor değil mi? Not, $x_1$ $A$ matrisinin birinci
kolonu. Kontrol edelim. Denklem

$$ \frac{du}{dt} = Au$$

$u$ için $e^{\lambda_1t}x_1$ koyalım, türev $t$'ye göre alındığına göre

$$ \lambda_1 e^{\lambda_1t}x_1 = A e^{\lambda_1t}x_1$$

İki taraftaki üstel değerler iptal olur, geri kalanlar

$$ \lambda_1 x_1 = A x_1$$

Bu da özdeğer/vektör formülüdür. Artık $u(t)$'nin son formülünü
yazabiliriz. 

$$ u(t) =
c_1 \cdot 1 \cdot 
\left[\begin{array}{c}
2 \\ 1
\end{array}\right]
+
c_2 e^{-3t} \cdot 
\left[\begin{array}{r}
1 \\ -1
\end{array}\right]
 $$

$t=0$ anında

$$ u(t) =
c_1 
\left[\begin{array}{c}
2 \\ 1
\end{array}\right]
+
c_2 
\left[\begin{array}{r}
1 \\ -1
\end{array}\right] = 
\left[\begin{array}{c}
1 \\ 0
\end{array}\right] 
 $$

İki tane formül, iki tane bilinmeyen var. Sonuç

$$ c_1 = \frac{1}{3},\ c_2 = \frac{1}{3}$$

yani

$$ u(t) =
\frac{1}{3} \left[\begin{array}{c} 2 \\ 1 \end{array}\right]
+
\frac{1}{3}  e^{-3t}  \left[\begin{array}{r} 1 \\ -1 \end{array}\right] 
 $$

Denklem kararlı konuma (steady-state) gelince, yani $t \rightarrow \infty$ için,
üstel bölüm yokolacaktır, ve baştaki terim kalacaktır. 

$$ u(\infty) =
\frac{1}{3}
\left[\begin{array}{c}
2 \\ 1
\end{array}\right]
 $$

Fakat üstteki durum, yani sabit bir istikrarlı konuma yaklaşmak her zaman
mümkün olmayabilir. Bazen sıfıra, bazen de sonsuzluğa da yaklaşabiliriz. 

1) Stabilite 

Ne zaman $u(t) \rightarrow 0$? Özdeğerler negatif ise, çünkü o zaman eksi
üstel değer olarak küçültücü etki yapacaklar. Eğer $\lambda$ kompleks bir
sayı olsaydı, mesela 

$$ e^{(-3 + 6i)} $$

Bu sayı ne kadar büyüktür? Mutlak değeri (absolute value) nedir? 

$$ | e^{(-3 + 6i)t}| = e^{-3t} $$

çünkü 

$$ |e^{6it}| = 1 $$

Niye 1? Çünkü kesin değer işareti içindeki $e^{6it} = \cos(6t)+i\sin(6t)$, ve
sağdaki $a+ib$ formudur, hatırlarsak kompleks eksenlerde $a$ ve $b$ üçgenin
iki kenarıdır, hipotenüs ise, üstte kesin değer olarak betimlenen şeydir,
uzunluğu $a^2 + b^2$, yani $\cos^2(6t) + \sin^2(6t) = 1$. Aslında $\cos$ ve
$\sin$ içine ne gelse sonuç değişmezdi. 

Yani kompleks kısım konumuz için önemli değil, çünkü nasıl olsa 1 olacak,
esas çözümü patlatabilecek (sonsuza götürecek), ya da küçültecek olan şey
reel kısım. Altta Re ile reel bölüm demek istiyoruz.

2) İstikrarlı konum: $\lambda_1 = 0$ ve öteki $Re \ \lambda < 0$. 

3) Patlama: herhangi bir $Re \ \lambda > 0$ ise. 

$\square$

Şu formülün matris formu nedir?

$$ u(0) =
c_1 
\left[\begin{array}{c}
2 \\ 1
\end{array}\right]
+
c_2 
\left[\begin{array}{c}
1 \\ -1
\end{array}\right] 
=
\left[\begin{array}{c}
1 \\ 0
\end{array}\right] 
$$

Şöyle

$$ 
\left[\begin{array}{cc}
2 & 1 \\ 1 & -1
\end{array}\right]
\left[\begin{array}{c}
c_1 \\ c_2
\end{array}\right] 
=
\left[\begin{array}{c}
1 \\ 0
\end{array}\right] 
 $$

Soldaki matris, özvektör matrisi $S$. O zaman $Sc = u_0$ 

$\square$

Probleme bakmanın değişik şekillerinden biri, onu ``bağlantısız
(decoupled)'' hale getirmek. Başlangıç formülüne dönersek

$$ \frac{du}{dt} = Au $$

Bu $A$ bağlantılı (coupled) bir halde. $u = Sv$ kullanırsak

$$ S\frac{dv}{dt} = ASv$$

$S$ özvektör matrisi. 

$$ \frac{dv}{dt} = S^{-1}ASv = \Lambda v$$

Bu geçişte iyi bilinen eşitlik $AS=S\Lambda$'nin $S^{-1}AS = \Lambda$
halini kullandık. Bu eşitliğin detayları için [1]'e bakabiliriz.

Böylece denklemler arasındaki bağlantıdan kurtulmuş olduk. Tüm sistem şu
hale geldi:

$$ \frac{dv_1}{dt} = \lambda_1 v_1$$

$$ \frac{dv_2}{dt} = \lambda_2 v_2$$

$$ .. $$

Bu sistemi çözmek daha kolay, her biri için ayrı ayrı çözümü yazalım,
mesela $v_1$

$$ v_1(t) = e^{\lambda_1 t} v_1(0) $$

Tamamı için

$$ v(t) = e^{\Lambda t} v(0) $$

Eğer $u=Sv$ ise o zaman $v=S^{-1}u$, ve $v(0)=S^{-1}u(0)$

$$ S^{-1}u(t) = e^{\Lambda t} S^{-1}u $$

$$ u(t) = Se^{\Lambda t}S^{-1} u(0) $$

Aradığımız formül üstteki şu bölüm

$$ e^{At} = Ae^{\Lambda t} S^{-1} $$

Ama bir matrisin üstel değer haline gelmesi ne demektir? Problemimizde elde
ettiğimiz çözüm bu öncelikle, yani $du/dt = Au$'nun çözümü $e^{At}$. Fakat,
yine soralım, bu demek? 

Güç serilerini (power series) hatırlayalım. $e^{x}$ için güç serisi nedir? 

$$ e^x = 1 + x + \frac{1}{2}x^2 + \frac{1}{6}x^3 + ... $$

1 yerine $I$ (birim matris), ve $x$ yerine $At$ alırsak

$$ e^{At} = I + At + \frac{(At)^2}{2} + \frac{(At)^3}{6} + ... +
 \frac{(At)^n}{n!} + ..
$$

Bu ``güzel'' bir Taylor serisi. Aslında matematikte iki tane çok güzel,
temiz Taylor serisi vardır, bir tanesi

$$ e^x = \sum_0^{\infty} \frac{x^n}{n!} $$

Öteki de Geometrik seri

$$ \frac{1}{1-x} = \sum_0^{\infty} x^n$$

Bu da en güzel güç serisidir. Onu da matris formunda kullanabilirdik aslında

$$ (I-At)^{-1} = I + At + (At)^2 + (At)^3 + ... $$

Bu formül, bu arada, eğer $t$ küçük bir değerse bir matrisin tersini
hesaplamanın iyi yaklaşıksal yöntemlerinden biri olabilir. Çünkü üstü
alınan küçük $t$ değerleri daha da küçülüyor demektir; üstteki terimlerin
çoğunu bir noktadan kesip atarız (yaklaşıksallık bu demek), ve sadece $I =
At + (At)^2$ hesabı yaparak matris tersini yaklaşıksal olarak
hesaplayabiliriz.

Zihin egzersizine devam edelim: $e^{At}$ açılımı $(I-At)^{-1}$ açılımından
hangisi daha iyi? İkincisi daha temiz, ama birincisi bir değere yakınsıyor
(converges), çünkü gitgide büyüyen $n$ değerleriyle bölüm yapıyorum. Demek
ki $t$ ne kadar büyürse büyüsün, toplam bir sonlu (finite) sayıya doğru
gidiyor. Fakat ikinci açılım öyle değil. Eğer $A$'nin 1'den büyük özdeğeri
var ise, toplam patlar ($At$ tüm özdeğerleri 1'den küçükse durum değişir
tabii, yani $|\lambda(At)| < 1$ ise).

Neyse, konumuza dönelim. $e^{At}$'nin $Se^{\Lambda t}S^{-1}$ ile bağlantılı
olduğunu nasıl görebilirim? $e^{At}$'yi anlamak için $S$ ve $\Lambda$'yi
anlamaya çalışmak lazım, $S$ zaten özvektör matrisi, $\Lambda$ ise köşegen
(diagonal) bir matris, tüm değerleri çaprazında, bunlar nispeten temiz
formlar, onları anlarsak işimiz kolaylaşacak.

Peki geçişi nasıl yapalım? $e^{At}$ açılımı olan güç serisini kullanarak bu
seriden $S$ ve $\Lambda$ çıkmasını sağlayabilir miyim acaba? Şunu zaten
biliyoruz: $A = S \Lambda S^{-1}$. Peki $A^2$ nedir?

$$ A^2 = (S \Lambda S^{-1})(S \Lambda S^{-1}) $$

$$ = S \Lambda S^{-1}S \Lambda S^{-1} $$

ortadaki $S$ ve $S^{-1}$ birbirini iptal eder. 

$$ = S \Lambda^2 S^{-1} $$

O zaman 

$$ e^{At} = I + S \Lambda S^{-1} t + \frac{S \Lambda^2 S^{-1}}{2}t^2 + ...$$

Dışarı tüm $S$'leri çıkartmak istiyorum. O zaman $I$'yi şu şekilde yazarsam
daha iyi olacak (ki içinden $S$'leri çekip çıkartabilelim), $I = SS^{-1}$

$$ = SS^{-1} + S \Lambda S^{-1} t + \frac{S \Lambda^2 S^{-1}}{2}t^2 + ...$$

ve

$$ = S (I + \Lambda t + \frac{\Lambda^2}{2}t^2 + ...) S^{-1}$$

Ortada kalanlar $e^{\Lambda t}$ değil mi? O zaman

$$ = S e^{\Lambda t}S^{-1} $$

Böylece geçişi yapmış olduk. Soru: bu formül hep işler mi? Hayır. Ne zaman
işler? Eğer $A$ köşegenleştirilebilen bir matris ise işler, yani içinden
$\Lambda$ matrisi çekip çıkartılabilecek matrisler için. Onu yapmanın ön
şartı ise $A$'nin tersine çevirelebilir olması, yani tüm özvektörlerinin
bağımsız olmasıdır.

Peki $e^{\Lambda t}$ nedir? Yani köşegen bir matris üstel değer olarak
karşımıza çıkınca sonuç ne olur? $\Lambda$ nedir?

$$ \Lambda =
\left[\begin{array}{ccc}
\lambda_1 && \\
&&.. \\
&& \lambda_n
\end{array}\right]
 $$

Bunu $e$ üzeri olarak hesaplayınca ne elde ederiz? Şunu elde ederiz:

$$ e^{\Lambda t} =
\left[\begin{array}{ccc}
e^{\lambda_1t} && \\
&&.. \\
&& e^{\lambda_n t}
\end{array}\right]
 $$

Not: Üstteki açılım sezgisel olarak tahmin edilebilecek bir şey olsa da,
üstel olarak bir matris olunca, beklenen her işlem yapılamayabiliyor. Mesela 
eğer matris

$$ 
\left[\begin{array}{cc}
a & b \\ c & d
\end{array}\right]
 $$

olsaydı o zaman her elemanı üstel olarak $e^{a}$, $e^b$ şeklinde kullanmak
ve matris içindeki yerine yazmak ise yaramazdı. Üstel matris dünyasının
kendine has bazı kuralları var.

Örnek

$$ y'' + by' + Ky = 0 $$

2. dereceden bu denklemi 1. dereceden formüllerden oluşan 2x2 bir
``sisteme'' dönüştürebiliriz. Ekstra bir denklem ortaya çıkaracağız, eğer
$y$ yerine bir vektör formundaki $u$'yu şu şekilde kullanırsak

$$ u = 
\left[\begin{array}{c}
y' \\ y
\end{array}\right]
 $$

$u$'nun türevi şöyle olur

$$ 
u' = 
\left[\begin{array}{c}
y'' \\ y'
\end{array}\right]
 $$

$u'$ formuna göre 2. derece denklemi şu şekilde temsil edebiliriz

$$ 
\left[\begin{array}{cc}
-b & -K \\
1 & 0
\end{array}\right]
\left[\begin{array}{c}
y' \\ y
\end{array}\right]
 $$

Genel olarak, mesela 5. dereceden bir denklemi alıp 5x5 boyutlarında 1. 
derece denklem sistemine geçmek te mümkündür. Bu geçiş alttaki matrisin 
üst satırına katsayı değerleri atayacak, ve onun altından başlayarak 1 
değerleri dolduracaktır. 

$$ 
\left[\begin{array}{rrrrr}
- & - & - & - & - \\
1 &&&& \\
& 1 &&& \\
&& 1 && \\
&&& 1 & 
\end{array}\right]
 $$

Alternatif Anlatım

%<a name='alt1'/>

$y(0)$ başlangıç şartına uygun $y' = Ay$ diferansiyel denklemini çözmek için
$A$'nin özvektör/değerleri kullanılabilir. Çözümler $y = e^{\lambda t} x$ ile
elde edilir ki $\lambda$ ve $x$, $A$'nin bir özdeğer/vektörüdür [2, sf. 349].

Burada temel gözlem bağımsız özvektörlerin bir baz oluşturabildiği ve bu
sebeple başlangıç değeri $y(0)$'i bu baz üzerinden temsil edebileceğimiz,

$$
y(0) = c_1 x_1 + ... + c_n x_n
\mlabel{1}
$$

ki $c_1,..,c_n$ bilmediğimiz katsayılar.

$y = e^{\lambda t} x$ bir çözümdür, kontrol edelim,

$$
\frac{\ud y}{\ud t} = A y
$$

icine $y = e^{\lambda t} x$ sokarsak,

$$
\frac{\ud }{\ud t} ( e^{\lambda t} x) =
\lambda e^{\lambda t} x =
A (e^{\lambda  t} x)
$$

$e^{\lambda  t}$  iptal olunca,

$$
Ax = \lambda x 
$$

ki üstteki özvektör/değer denklemi. Şimdi çözümü

$$
y(t) = c_1 e^{\lambda_1 t} x_1 + ... + e^{\lambda_n t} x_n
$$

olarak gösteririz, dikkat edersek $t=0$ anında üstteki denklem (1) ile aynı.
Özvektör ve değerleri bulduktan sonra bu başlangıç şartını kullanarak
$c_1,..,c_n$ değerlerini bulmak mümkün olacak.

Ondan önce niye çözümlerin lineer kombinasyonlarının da bir çözüm olduğunu
görelim. Sistemi $x' = Ax$ olarak tanımlayalım, ki $x = [x_1, x_2]$.  İspat
matris çarpımın, ve türevin lineer operatör olmalarıyla alakalı, eğer

$$
y(t) = c_1 x_1(t) + c_2 x_2(t)
$$

ise,

$$
y'(t) = (c_1 x_1(t) + c_2 x_2(t))' = c_1 x_1'(t) + c_2 x_2'(t)
$$

olduğunu biliyoruz. Matris çarpımı da aynı şekilde lineer,

$$
A y(t) = A(c_1 x_1(t) + c_2 x_2(t)) = c_1 A x_1(t) + c_2 A x_2(t)
$$

Daha önceden biliyoruz ki $x_1,x_2$'in çözüm olması $x_1'=Aax_1$ ve $x_2' = A
x_2$ anlamina geliyor. Bu formullerin birincisini $c_1$, ikincisini $c_2$ ile
çarpıp toplarsak,

$$
c_1 x_1'(t) + c_2 x_2'(t) = c_1 A x_1(t) + c_2 A x_2(t)
$$

Yani $y'(t) = A y(t)$ elde etmiş olduk, yani yine $A$ bazlı bir diferansiyel
denklem, o zaman $x_1,x_2$'nin tüm lineer kombinasyonları $x' = Ax$ için de bir
çözümdür.

Şimdi bir örnek problemde görelim.

Örnek

$$
y' = \left[\begin{array}{rrr}
-2 & 1 \\ 1 & -2
\end{array}\right]
y, \qquad
y(0) = \left[\begin{array}{c} 6 \\ 2 \end{array}\right]
$$

denkleminin çözümlerini bul, hangi çözüm $y(0)$'a karşılık geliyor?

Çözüm

$A$'nin özdeğerlerini bulalım,

\begin{minted}[fontsize=\footnotesize]{python}
import numpy.linalg as lin
A = np.array([[-2,1],[1,-2]])
evals,evec = lin.eig(A)
print (evals[0], evec[:,0])
print (evals[1], evec[:,1])
\end{minted}

\begin{verbatim}
-1.0 [0.70710678 0.70710678]
-3.0 [-0.70710678  0.70710678]
\end{verbatim}

Bunlar normalize edilmiş özvektörler bu arada, aslında

\begin{minted}[fontsize=\footnotesize]{python}
print (evec / 0.70710678)
\end{minted}

\begin{verbatim}
[[ 1. -1.]
 [ 1.  1.]]
\end{verbatim}

ile kitap sonucunu elde edebiliriz. 

$c_1,c_2$ değerlerini bulmak için $c$ vektörü yaratalım, özvektörler $x_1,x_2$'i
bir $V$ matrisinin kolonlarına koyalım, böylece $y(0) = c_1 x_1 + ... + c_n x_n$
yerine $y(0) = Vc$ demek mümkün olacak, ve bilinmeyen $c$ için çözebiliriz,

\begin{minted}[fontsize=\footnotesize]{python}
V = evec / 0.70710678 
print (V.T) # kolonlarda ozvektorler icin devrik aldik
soln = lin.solve(V.T, np.array([[6,2]]).T)
print (soln)
\end{minted}

\begin{verbatim}
[[ 1.  1.]
 [-1.  1.]]
[[2.        ]
 [3.99999999]]
\end{verbatim}

$y(0)$ icin kontrol edersek,

$$
\left[\begin{array}{r}
6 \\ 2
\end{array}\right] =
2 
\left[\begin{array}{r}
1 \\ -1
\end{array}\right] +
4 
\left[\begin{array}{r}
1 \\ 1
\end{array}\right] 
$$

eşitliği doğru. 

O zaman nihai sonuc

$$
y(t) =
4 e^{-t}
\left[\begin{array}{r}
1 \\ 1
\end{array}\right]  +
2 e^{-3t}
\left[\begin{array}{r}
1 \\ 1
\end{array}\right] =
\left[\begin{array}{r}
4 e^{-t} + 2 e^{-3t} \\
4 e^{-t} + 2 e^{-3t}
\end{array}\right]
$$

Kaynaklar

[1] Bayramlı, Hesapsal Bilim, {\em Ders 6}

[2] Strang, {\em Differential Equations and Linear Algebra}

\end{document}

