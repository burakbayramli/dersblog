\documentclass[12pt,fleqn]{article}\usepackage{../../common}
\begin{document}
İdeal Gazlar Kanunu (İdeal Gas Law)

Önce bazı terimler. Bir mol (mole) terimi mesela, mol önceden belirli bir
molekül sayısıdır. Tutarlı olması için herkesin kabul ettiği bir sayı, özel bir
temel parçacığa bağlanmış, bir mol 12 gramlık karbon-12 içindeki atom sayısı
[2, sf. 550]. Mole molekül sayısı aynı zamanda ünlü Avagadro sabitidir,
$N_A$ ile gösterilir,

$$
N_A = 6.02 x 10^{23} mol^{-1}
$$

Yani bir mol içinde üstteki kadar molekül var. Bir materyal içinde kaç mol var
hesabı için $n = N / N_A$ kullanabiliriz, $N$ tüm molekül sayısı, $N_A$ bir mol
içindeki molekül sayısı, bölüm bize istenen sonucu verir.

Basınç

Peki mikro etkileşimlerden yola çıkarak basınç kavramını türetebilir miyiz
acaba? 19'uncu yüzyıl sonlarına doğru bu başarıldı ve gerçekten temel mekanik
kanunlarının basit bir model üzerinen makro açıklamalar yapabilmesinin çok güzel
bir örneği.

Basıncın gaz moleküllerinin bir yüzeye çarpmasından ortaya çıktığını
hatırlayalım. Bu kuvvet tabii ki Newton kanunundan hareketle,

$$
f = m a = m \frac{\ud v}{\ud t}
$$

Hız $v$'ye molekül içinde olduğu kabin / yüzey duvarına çarptığında ona dik olan
hız diyelim [1]. Bu türevi hesaplamak için, ki birim zamanda hız değişimi
gerekiyor, kenarları $L$ uzunluğunda bir küp içinde tek bir gaz molekül olduğunu
düşünelim.

Basitleştirme amacıyla diyelim ki bu molekül sürekli küp kutu içinde ileri geri
gidip geliyor, bir duvara çarpınca bir süre sonra geri geliyor. Bu molekül bir
duvara çarptığında $v$ hızında çarptığında (yani $mv$ momentumuyla) elastik
olarak geri sekecektir, ve $-v$ ile tam ters yöne geri gitmeye başlayacaktır.

\includegraphics[width=20em]{phy_005_basics_04.png}

O zaman her çarpışma için hız değişimi $2v$, momentum değişimi ise $2mv$
olur.

Tabii aslında eğer daha genel formülize etmek gerekirse bu çarpışma sırasında
$\bar{v}$ hızının duvara dik olan bileşeni $v_x$'yi düşünüyoruz.

\includegraphics[width=20em]{phy_005_basics_05.png}

Yani momentum değişimi

$$
\Delta p_x = (-m v_x) - (m v_x) = - 2 m v_x 
$$

Demek ki duvara transfer edilen momentum $2 m v_x$. 

Birim zaman $\Delta t$'ye bir molekün iki çarpışma arasında geçen zaman dersek,
ve $v_x$ hızında $2L$ yol katedilmişse, $\Delta t  = 2 L / v_x$ demektir, ve

$$
F = \frac{\Delta p_x}{\Delta t} = \frac{2 m v_x}{2 L / v_x} = \frac{m v_x^2}{L}
$$

Basınç birim alana uygulanan kuvvettir, ve küpün bir kenarının $L^2$ alanında
olduğunu düşünürsek, 

$$
P = \frac{m v^2}{L^3} = \frac{m v^2}{V}
$$

$V$'yi kutunun hacmi olarak aldık, ve $V = L^3$.

Birden fazla molekülü düşünmek istiyoruz şimdi, mesela bir averaj
üzerinden.. Fakat her molekül hem negatif hem pozitif yönde aşağı yukarı aynı
miktarda hareket yapar (rasgele hareket olduğu için) ve bu tür bir hareket
üzerinden averaj almak bizi sıfır değerine götürür. Bu sebeple ortalamasını
almadan önce hızların karesini almak istiyoruz,

$$
\bar{v^2} = \frac{v_1^2 + v_2^2 + ... + v_N^2 }{N} = \frac{\sum_i v_i^2}{N}
$$

ve ortalama değeri bulmak için $\sqrt{\bar{v^2}}$ kullanıyoruz. Bu hesaba kök
kare ortalaması (root mean square -RMS-) ismi de verilir. Şimdi tüm $N$
moleküller üzerinden bir basınç hesaplamak istersek, $N$ tane molekül, ama belli
bir anda sadece Kartezyen kordinat sisteminde sadece üç yönden sadece biri
yönünde etki var, o zaman $N$ ile çarpıp 3'e bölmek lazım, 

$$
P = \frac{N}{3} \frac{m \bar{v^2}}{V}
$$

Bu formül içinde bir kinetik enerji formülasyonu görülebiliyor, averaj kinetik
enerjiye $\epsilon = m \bar{v^2} / 2$ dersek, üstteki formülü

$$
PV = \frac{N}{3} m \bar{v^2} = \frac{2}{3} N \epsilon
$$

olarak yazabiliriz.

Eğer bu formülü sıcaklık içerek şekilde değiştirmek istiyorsak; biliyoruz ki
sisteme eklenen her Joule enerji ve bir derece sıcaklık değişimi arasındaki
ilişkiyi $k$ sabiti kontrol eder [5, 29-16] bu sabit $k = 1.38 x 10^{23}$ Joule
/ Kelvin'dir, o zaman enerjiden sıcaklığa geçiş için $kT$ kullanabiliriz, hatta
bir $3/2$ eklenerek üstteki 2/3 iptali amaçlanır,

$$
\epsilon = \frac{3}{2} k T
$$

Ve,

$$
PV = \left( \frac{2}{3} N \right) \left( \frac{3}{2} k T \right) = N k T
$$

Devam edelim, $n = N / N_A$ olduğunu da biliyoruz ki $N_A = 6.02 x 10^{23}$,
Avagadro'nun sayısı, $n$ örneklemdeki mol sayısı, $N$ ise örneklemdeki tüm
moleküller [2, sf. 550],

$$
PV = n N_A k T
$$

Tabii bu bizi $R$ denen bir diğer sabite götürüyor, $R = 8.31 J/mol \cdot
K$. Onun $k$ ve $N_A$ ile ilişkisi şöyle,

$$
k = \frac{R}{N_A}
$$

O zaman,

$$
PV = n R T
$$

İdeal gazlar kanununa erişmiş olduk.


Mol sayısını şöyle ifade edebilirdik,

$$
n = m / w_m
$$

ki $m$ gazın kütlesi, $w_m$ ise moleküler ağırlık. Bunları birbirine bölünce
doğal olarak mol sayısı ortaya çıkar. Ama $n$'yi iki üstteki formüle koyunca,

$$
PV = \frac{m}{w_m} R T
$$

Bir düzenleme yaparsak,

$$
P = \frac{m}{V} \frac{R}{w_m} T
$$

$m/V$ yoğunluk, ona $\rho$ diyebiliriz, $R / w_m$ ise evrensel gaz
sabiti $R$'nin gazın moleküler ağırlığına bölünmüş hali, ona
yeni bir sabit $R_m$ diyebiliriz, o zaman daha öz

$$
P = \rho R_m T
$$

formülü elde edilir.

İç Enerji (Internal Energy)

Daha önce görmüştük, tek atomun ortalama hareketsel kinetik enerjisi
sıcaklığa bağlıydı, ona daha önce $\epsilon$ demiştik, şimdi $K_{avg}$
diyelim, ki $K_{avg} = 3/2 k T$. O zaman, ve $n$ mol miktarında bir
örneklemin içinde $n N_A$ molekül olacağı için, örneklemin iç
enerjisi $E_{int}$ şöyle hesaplanabilir [2, sf. 564],

$$
E_{int} = (n N_A) K_{avg} = (n N_A) (\frac{3}{2} k T)
$$

Daha önce gördük $k = R/N_A$, üstteki $E_{int}$ içine koyarsak,

$$
E_{int} = \frac{3}{2} n R T
$$

Mol Spesifik Isı (Molar Specific Heat of a Gas)

Bir gazın sıcaklık artışına tekabül eden ısı enerjisi (girdisi) faydalı
olabilecek bir büyüklüktür, fakat aynı sıcaklık değişimine giden ayrı yollar,
farklı ısı hesaplarına sebep verebilir. Değişimin en çok ortaya çıkan iki
versiyonu için farklı bir mol spesifik ısı öne sürmek daha iyi olur, bunlardan
biri basınç sabit tutulduğu durumdaki, diğeri de hacim sabit tutulduğu
durumdaki mol spesifik ısı.

$Q = n C_P \Delta T$

$Q = n C_V \Delta T$

Üstteki $C_V$ sabit hacimdeki mol spesifik ısı, $C_P$ sabit basınçtaki.

Sabit hacim durumu icin $Q = \Delta E_{int}$, o zaman 

$\Delta E_{int} = n C_V \Delta T$

Eger sicaklikta degisim yoksa 

$E_{int} = n C_V T$

Bu denklem tum ideal gazlar icin gecerlidir, ideal gaz derken molekulunde birden
fazla atom olan gazlar icin.

\includegraphics[width=15em]{phy_005_basics_06.png}

Kaynaklar

[1] Chang, {\em Physical Chemistry for the Biosciences},
    \url{https://chem.libretexts.org/@go/page/41408}

[2] Resnick, Fundamentals of Physics, 10th Ed

[5] Feynman, {\em Feynman Lectures on Physics, I}

\end{document}
