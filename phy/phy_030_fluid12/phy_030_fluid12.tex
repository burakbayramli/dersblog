\documentclass[12pt,fleqn]{article}\usepackage{../../common}
\begin{document}
Gazlar, Sıvılar - 2

Muhafaza Kanunları Tek Boyut

İçinde gaz olan sadece tek boyutuna baktığımız bir tüp düşünelim, $x$ tüpün
üzerindeki bir noktayı temsil edecek, $\rho(x,t)$ ise tüpün $x$ noktasında ve
$t$ anındaki yoğunluğunu verecek diyelim. Yoğunluğu kullanarak $x_1$ ve $x_2$
noktaları arasındaki $t$ anındaki kütle

$$
\int _{x_1}^{x_2} \rho(x,t) \ud x
$$

ile hesaplanabilir. Tüpün duvarları tam izole ise ve kütle yoktan varedilip
yokedilemeyeceğine göre tüpe gaz giriş ya da çıkış sadece $x_1,x_2$
noktalarından olabilir [4, sf. 14]. Şimdi bir gaz hareket hızı düşünelim,
$u(x,t)$ ile, o zaman gaz akma oranı, ya da akış (flux)

$$
flux = \rho(x,t) u(x,t)
$$

olur. Üstteki fiziksel kurallardan hareketle $[x_1,x_2]$ deki kütlenin
değişim oranı $x_1$ ve $x_2$ noktalarındaki akışın farkına eşit olmalıdır,

$$
\frac{\ud}{\ud t} \int _{x_1}^{x_2} \rho(x,t) \ud x =
\rho(x_1,t) u(x_1,t) - \rho(x_2,t) u(x_2,t)
$$

İşte bu muhafaza kanununun entegral formudur. 

Üstteki formülü $t_1,t_2$ zaman aralığı için entegre edersek, ki böylece
bu zaman içindeki tüm toplam akışı hesaplayabilelim, o zaman

$$
\int_{t_1}^{t_2} \left( \frac{\ud}{\ud t} \int _{x_1}^{x_2} \rho(x,t) \ud x  \right)  =
\int_{t_1}^{t_2} \rho(x_1,t) u(x_1,t) \ud t -
\int_{t_1}^{t_2} \rho(x_2,t) u(x_2,t) \ud t
$$

Soldaki kısım zaman üzerinden türevin yine zaman üzerinden entegrali, o zaman
yokolabilir, Calculus'un Temel Teorisi üzerinden basitleştirirsek,

$$
\int _{x_1}^{x_2} \rho(x,t_2) \ud x -
\int _{x_1}^{x_2} \rho(x,t_1) \ud x  = 
\int_{t_1}^{t_2} \rho(x_1,t) u(x_1,t) \ud t -
\int_{t_1}^{t_2}  \rho(x_2,t) u(x_2,t) \ud t
$$

Ufak bir yer değiştirme sonrası

$$
\int _{x_1}^{x_2} \rho(x,t_2) \ud x =
\int _{x_1}^{x_2} \rho(x,t_1) \ud x  +
\int_{t_1}^{t_2} \rho(x_1,t) u(x_1,t) \ud t -
\int_{t_1}^{t_2}  \rho(x_2,t) u(x_2,t) \ud t
$$

Üstteki formun değişik bir şekli ileride lazım olacak, zaman adımı atmaya
uğraştığımız hesapsal yöntemlerde $t_1$ ve $t_2$ üzerinden bir entegral, hesabı
bir sonraki zamana geçirmeye uğraştığımızda, adım attığımızda.

Neyse şimdi diferansiyel forma geçise dönelim. Bu noktada $\rho(x,t)$ ve
$u(x,t)$'nin türevi alınabilir fonksiyonlar olduğunu farz ediyoruz. Üstekini,
yine ufak bir değişim sonrası,

$$
\int _{x_1}^{x_2} \rho(x,t_1) \ud x  +
\int _{x_1}^{x_2} \rho(x,t_2) \ud x -
\int_{t_1}^{t_2}  \rho(x_2,t) u(x_2,t) \ud t -
\int_{t_1}^{t_2} \rho(x_1,t) u(x_1,t) \ud t = 0
\mlabel{4}
$$

olarak görelim. Eğer Calculus'un Temel Teorisi ile ilk iki terime
$\int_{t_1}^{t_2} .. \ud / \ud t$ son iki terime $\int_{x_1}^{x_2} .. \ud / \ud x$
ekleyebilirsek, tüm terimlerde aynı entegraller olacağı için, 
$\int_{t_1}^{t_2} \int_{x_1}^{x_2} $ altında tüm terimleri gruplayıp
basitleştirmek mümkün, ve bunlar sıfıra eşit olur. Bu bizi diferansiyel
forma götürebilir. Yani

$$
\rho(x,t_2) - \rho(x,t_1) = \int_{t_1}^{t_2}
\frac{\partial }{\partial t} \rho(x,t) \ud t
$$

ve

$$
\rho(x_2,t)u(x_2,t) - \rho(x_1,t)u(x_1,t) =
\int _{x_1}^{x_2} \frac{\partial }{\partial x} (\rho(x,t)u(x,t)) \ud x
$$

eşitliklerinden hareketle, bunları (4)'e uygulayıp

$$
\int _{t_1}^{t_2} \int _{x_1}^{x_2}  \left\{
\frac{\partial }{\partial t} \rho(x,t)  +
\frac{\partial }{\partial x} (\rho(x,t)u(x,t))
\right\} \ud x \ud t = 0
\mlabel{5}
$$

elde ediyoruz. Bu ifadenin $[x_1,x_2]$ ve $[t_1,t_2]$ arasındaki tüm değerlerde
doğru olması gerektiği için entegre edilenin sıfır olması gerekiyor ([5]'dekine
benzer bir mantık yürütüldü), yani

$$
\rho_t + (\rho v)_x = 0
$$

olmalı. Böylece kütlenin muhafaza kuralını diferansiyel formda elde etmiş olduk.

Bu formu izole halde çözmenin tek yolu $v$'nin önceden bilindiği durumdadır, ya
da $v$ fonksiyon $\rho(x,t)$'ye bağlı bir fonksiyon olmalıdır, yani
$f(\rho) = \rho v$ gibi. Bu durumda üstteki ifade $\rho$ için tek sayısal
muhafaza kanunu haline gelir,

$$
\rho_t + f(\rho)_x = 0
$$

Diğer Muhafaza Edilen Büyüklükler

Tekrar üzerinden geçip, genişletelim, [6, sf. 15] notasyonu ile devam edelim,
ölçmek istediğimiz sıkıştırılamayan bir sıvı, gaz büyüklüğü var, ve bunu $x$
noktasında $t$ anı için $q(x,t)$ ile ölçüyoruz, takip ediyoruz. Mesela sıvı
içine bir işaretleyici karıştırılmış, mürekkep gibi, o takip ediliyor. Takip
edilen bu ölçümün yoğunluğu $q(x,t)$ olsun, bu fonksiyonun ne olduğunu anlamak
istiyoruz. Bu işaretleyicinin $x_1$ ve $x_1$ arasındaki kütlesinin hesabı için

$$
\int _{x_1}^{x_2} q(x,t) \ud x
$$

hesaplanır. Şimdi akış (flux) kavramını tekrar tanıştıralım, herhangi bir $x$
noktasında ve $t$ anındaki işaretleyici yoğunluğunun akma oranı akistir.
Bilinen $u(x,t)$ hızı ve yoğunluk $q(x,t)$'yi çarparak onu elde edebiliriz,

$(x,t)$'deki akış  = $u(x,t) q(x,t)$

Dikkat, akış sıfırdan büyükse bu sağa doğru akış demektir, küçükse sola doğru
akış demektir. Hız bilinen bir büyüklük olduğuna göre bir akışı $f$'yi $q$'nun
fonksiyonu olarak yazabiliriz,

akış = $f(q,x,t) = u(x,t) q$ 

Şimdi üstteki entegral ile akış formülünü bağlayalım. Kütle muhafaza edildiği
için $x_1$ ve $x_2$ arasındaki kütleyi hesaplamıştık hatırlarsak, bu kütlenin
zamana göre değişim oranı sadece ve sadece o bölgeye sağdan ve solda olacak
akışlar ile mümkündür.

$$
\frac{\ud}{\ud t} \int _{x_1}^{x_2} q(x,t) \ud x =
f(q(x_1,t)) - f(q(x_2,t)) 
$$

Dikkat, $x_2$ üzerindeki akışta eksi işareti var çünkü sağdaki sınırdan
sola doğru akışı istiyoruz, ve $x_1$ üzerindeki akışta artı işaret var,
çünkü o sınırdan sağa doğru giden, $[x_1,x_2]$ bölgesine giren akışa
bakıyoruz.

Üstteki formülün sağ tarafını Calculus'un standart notasyonu ile yazabiliriz,

$$
\frac{\ud}{\ud t} \int _{x_1}^{x_2} q(x,t) \ud x =
-f(q(x_1,t)) \biggr\rvert_{x_1}^{x_2}
$$

Calculus'a geçtiğimize göre sağ tarafı Calculus'un Temel Teorisi üzerinden
türevin entegrali haline çevirebiliriz,

$$
\frac{\ud}{\ud t} \int _{x_1}^{x_2} q(x,t) \ud x =
- \int_{x_1}^{x_2} \frac{\partial }{\partial x} f(q(x,t)) \ud x
$$

Şimdi zaman türevini entegral içine alabiliriz. Ayrıca eşitliğin sol ve sağ
kısmı aynı entegrale sahip oldukları için onları birleştirmek mümkün,

$$
\int _{x_1}^{x_2}
\left[
\frac{\ud}{\ud t} q(x,t) + \frac{\partial }{\partial x} f(q(x,t)) 
\right]
\ud x  = 0
$$

Daha önce (5) formülü için kullandığımız mantık geçerli, o zaman entegre
edilen sıfır olmalı, böylece alttaki diferansiyel denklemi elde ediyoruz,

$$
\frac{\ud}{\ud t} q(x,t) + \frac{\partial }{\partial x} f(q(x,t))  = 0
$$

Ya da

$$
q_t(x,t) + f(q(x,t))_x  = 0
$$

Aynen kütle muhafaza edildiği gibi momentum da muhafaza edilebilir. Bu durumda
$\rho(x,t) u(x,t)$ bir momentum yoğunluğu verir, ki $\rho$ kütle yoğunluğu, ve
$\rho u$ çarpımının iki nokta arasındaki entegrali o aralıktaki toplam momentumu
hesaplar, ve bu toplam sadece o aralığa sınırlardan girecek hareket eden sıvıyla
gelecek dış momentumlar ile değisebilir. Eğer $q = \rho u$ ise akış
$(\rho u) u = \rho u^2$ ile hesaplanır.

Fakat momentum hesabına etki eden başka faktörler de var. Üstteki makroskopik,
büyük ölçekteki bir etkiydi. Mikroskopik bir etki de var. Çünkü düşünürsek eğer
gaz hiç hareket etmiyor bile olsaydı, yani makroskopik görünen hız $u=0$
olsaydı, hala gaz içindeki moleküller hareket halinde olurdu [6, sf 292]. Öyle
değil mi?  Eğer gaz ısısı mutlak sıfır üzerinde ise bir hareket var
demektir. İşte bu hareketlilik gaz içinde basınç yaratır. Herhangi bir $x_1$
noktasındaki basıncı anlamak için tek boyutlu tüpümüzün o noktasına bir hayali
duvar soktuğumuzu düşünelim ve bu duvarın her iki tarafına gaz tarafından
uygulanacak kuvveti (birim alan bazlı olarak) hesaplayalım. Bu kuvvetler
normalde aynı mutlak büyüklükte ama ters işaretli olurlar. Fakat tüpün her
iki ucunu göz önüne alırsak eğer bu iki uçta basınç farkı var ise bu
iç titreşimlerin bir tarafta diğerine göre daha fazla olduğu anlamına gelir
ve bu fark bizim baktığımı tüp aralığına momentum eklenmesi olarak yansır.

O zaman momentum akışını $\rho u^2 + p$ olarak hesaplamak gerekir, entegral
muhafaza kanunu olarak,

$$
\frac{\ud}{\ud t} \int _{x_1}^{x_2} 
\rho(x,t) u(x,t) \ud x = - [\rho u^2 + p] _{x_1}^{x_2}
$$

Dikkat $[ ... ]_{x_1}^{x_2}$ işlemi ile iki uç arasındaki basınç farkını
formüle katmış oluyoruz.

Ve tekrar daha önce gördüğümüz matematiksel işlemleri yine uygularsak,
$\rho,u,p$'nin pürüzsüz fonksiyonlar olduğunu varsayarak momentum denkleminin
diferansiyel formunu elde edebiliriz,

$$
(\rho u)_t + (\rho u^2 + p)_x = 0
$$

Energy için de benzer bir taşınma formülü mümkün. $E$ sembolüyle birim hacimdeki
enerji yoğunluğunu temsil edelim, bu enerji de gaz akışı içinde taşınacaktır, bu
durum makroskopik akış terimi $E u$ sonucunu verir. Ayrıca mikroskopik seviyede
basınç $p$'nin yarattığı da kinetik enerjide bir $pu$ akışına sebep olacaktır.
Enerji denklemini o zaman,

$$
E_t + [(E + p) u ]_x = 0
$$

ile gösteririz. $E$ ile $p$'nin toplanmış olması garip gelebilir, birisi enerji
diğeri kuvvet. Fakat ölçüm birimlerini kontrol etmek gerekirse, $E$ içinde birim
hacimdeki enerji tutuluyor, mesela $m^3$ içindeki enerji $N m$, basınç ise tanım
itibariyle birim alandaki kuvvettir, yani $N / m^2$. $E$ için o zaman
$N m / m^3 = N / m^2$ elde edilir ki bu basıncın birimi ile aynıdır.

Tüm bu denklemleri biraraya koyunca, Euler Gaz Dinamiği formüller (Euler
Equation of Dynamics) elde edilir.

$$
\left[\begin{array}{c}
\rho \\ \rho u  \\ E
\end{array}\right]_t
+
\left[\begin{array}{c}
\rho u \\ \rho u^2 + p \\ (E+p) u 
\end{array}\right]_x 
= 0
$$

Fakat dikkat edersek bu noktada elimizde dört tane değişken var, ama sadece üç
tane muhafaza kanunu listelendi. Bu sistemi ``kapatmak'' için yani dört
bilinmeyen için dört denkleme sahip olmak için bir tane daha denkleme
ihtiyacımız var. Bu denklem [3]'te işlenen ideal gazların konum formülü
olabilir, bu formül basıncın diğer büyüklüklere nasıl bağlı olduğunu
gösterir, biz politropik duruma odaklanacağız, bu durumda formül

$$
p = \rho e (\gamma - 1)
$$

ki [3]'te görüldüğü gibi $\gamma$ spesifik ısıların oranıdır.

Üstteki formül birazdan görülecek temel değişken formuna geçişte kullanılacak.

Temel Değişkenler

Euler formüllerinin belki daha rahat anlaşılacak formu öz / ilkel / temel
(primitive) formülasyonu diye bilinir ve sadece $\rho,u,p$ değişkenlerini baz
alır, yani $t$ ve $x$ türevi alınan değişken vektörü bu üç temel değişkeni
içerecektir. Bunun için biraz cebirsel manipulasyon
başlarsak,

$$
\rho_t + (\rho u)_x = 0 
$$

$$
\implies \rho_t + \rho_x u + \rho u_x = 0
\mlabel{3}
$$

Öyle değil mi? Tek yaptığımız parantez içindeki türevi açmak oldu. 

İkinci denklem için de benzer işlemi yapabiliriz,

$$
(\rho u)_t = (\rho uu + p)_x
$$

$$
= \rho_t u + \rho u_t + \rho u u_x + u (\rho u)_x + p_x
$$

Birinci ve dördüncü terimleri dikkat edersek, onlar aslında (sıfıra eşit)
yoğunluk diferansiyel formülünün $u$ ile çarpılmış hali değil mi?

$$
= \cancel{u (\rho_t + (\rho u)_x )} + \rho u_t + \rho u u_x + p_x
$$

Parantez içi sıfır olduğu için orası iptal oldu, geriye kalanları $\rho$
ile bölersek,

$$
u_t + u u_x + \frac{1}{\rho} p_x = 0
\mlabel{2}
$$

Enerji
denklemine gelelim; enerji çoğunlukla

$$
E = \rho e + \frac{1}{2} \rho u^2
\mlabel{1}
$$

şeklinde ayrıstılır, ki $e$ iç enerji, $\frac{1}{2}\rho u^2$ ise kinetik
enerjidir. Değişken $e$ birim kütle bazlı iç enerji. İç enerji yer değişimsel,
dönüşsel, titreşim, vb. formdaki pek çok enerji türünü temsil eder [6, sf. 293].
Politropik ideal gazların konum formülünden $e$ tanımını alıp kullanırsak,

$$
E = \frac{p}{\gamma - 1} + \frac{1}{2} \rho u^2
$$

Üstteki form [6] bazlı, biz [7] bazlı olarak $E$'için $\rho$ ile çarpılmamış
hali baz alacağız, böylece kısmi türev $\rho E$ üzerinden alınacak, yani

$$
E = e + \frac{1}{2} u^2
$$

Euler denklemindeki muhafazakar enerji formu da

$$
\frac{\partial (\rho E)}{\partial t} + \frac{\partial }{\partial x} (\rho u E + up) = 0
$$

oluyor. Buradan devam edersek, türevleri açalım,

$$
\frac{\partial \rho E}{\partial t} + \frac{\partial \rho u E}{\partial x} +
\frac{\partial }{\partial x} (up) = 0
$$

$$
\rho \frac{\partial E}{\partial t} + E \frac{\partial \rho}{\partial t} +
\rho u \frac{\partial E}{\partial t} + E \frac{\partial (\rho u)}{\partial x} +
\frac{\partial }{\partial x} (up) = 0
$$

Üstte ikinci ve dördüncü terim birarada gruplama sonrası süreklilik denklemini
verir,

$$
\rho \frac{\partial E}{\partial t} +
E ( \cancel{\frac{\partial \rho}{\partial t} + \frac{\partial (\rho u)}{\partial x}}) +
\rho u \frac{\partial E}{\partial x} +
\frac{\partial }{\partial x} (up) = 0
$$

$$
\rho \frac{\partial E}{\partial t} +
\rho u \frac{\partial E}{\partial x} +
\frac{\partial }{\partial x} (up) = 0
$$

Şimdi $E = e + \frac{1}{2} u^2$ kullanalım, ve üstte yerine koyarak türevi
açalım,

$$
\rho \frac{\partial e}{\partial t} + \frac{1}{2} \rho \frac{\partial u^2}{\partial t}+
\rho u \frac{\partial e}{\partial x} + \frac{1}{2} \rho u \frac{\partial u^2}{\partial x}+
\frac{\partial }{\partial x} (up) = 0
$$

$$
\rho \frac{\partial e}{\partial t} +
\rho u \frac{\partial u}{\partial t} +
\rho u \frac{\partial e}{\partial x} +
\rho u^2 \frac{\partial u}{\partial x} +
\frac{\partial }{\partial x}(up) = 0
$$

Üstteki $u_t$ için (2)'deki denklemi baz alarak bir eşitlik ortaya
çıkartabiliriz, yani

$$
u_t = -u u_x - \frac{1}{\rho} p_x
$$

Bunu iki üste sokalım,

$$
\rho \frac{\partial e}{\partial t} +
\rho u \left[
  \cancel{-u \frac{\partial u}{\partial x}} - \frac{1}{\rho} \frac{\partial p}{\partial x}
\right] +
\rho u \frac{\partial e}{\partial x} +
\cancel{\rho u^2 \frac{\partial u}{\partial x}} +
\frac{\partial }{\partial x} (up) = 0
$$

$$
\rho \frac{\partial e}{\partial t} -
u \frac{\partial p}{\partial x} +
\rho u \frac{\partial e}{\partial x} +
\frac{\partial }{\partial x} (up) = 0
$$

İkinci ve dördüncü terimleri de basitleştirebiliriz, çünkü

$$
\frac{\partial }{\partial x} (up) =
u\frac{\partial p}{\partial x} + 
p\frac{\partial u}{\partial x}  
$$

$$
\implies
\frac{\partial }{\partial x} (up) - u\frac{\partial p}{\partial x} =
p\frac{\partial u}{\partial x}
$$

Ana denklemde yerine koyarsak iç enerji denklemini elde ediyoruz,

$$
\rho \frac{\partial e}{\partial t} +
\rho u \frac{\partial e}{\partial x} +
p \frac{\partial u}{\partial x} = 0
$$

$\rho$ ile bölelim,

$$
\frac{\partial e}{\partial t} +
u \frac{\partial e}{\partial x} +
\frac{p}{\rho} \frac{\partial u}{\partial x} = 0
$$

Şimdi $e$ için [3]'de işlenen politropik ideal gazların konum formülünü koyalım,
ama dikkat, üstteki türetimde $\rho$ ile çarpılmamış hali baz aldık, yani

$$
e = \frac{p}{\rho (\gamma - 1)}
$$

$$
\frac{\partial }{\partial t} \left(\frac{p}{\rho}\right) +
u \frac{\partial }{\partial x} \left(\frac{p}{\rho} \right) +
(\gamma - 1)\frac{p}{\rho} \frac{\partial u}{\partial x} = 0
$$

Buna nasıl eriştik görülüyor herhalde; $e$ içindeki $1/(\gamma-1)$ sabit
olduğu için türev dışına çıkacaktı, türev sonrası tüm terimleri $(\gamma-1)$
ile çarparsak üstteki sonuca ulaşabiliyoruz.

Devam edelim, üstteki türevleri açalım,

$$
-\frac{p}{\rho^2} \frac{\partial \rho}{\partial t} +
\frac{1}{\rho} \frac{\partial p}{\partial t} -
\frac{up}{\rho^2} \frac{\partial \rho}{\partial x} +
\frac{u}{\rho} \frac{\partial p}{\partial x} +
(\gamma - 1)\frac{p}{\rho} \frac{\partial u}{\partial x} = 0
$$

Birinci ve üçüncü terimleri gruplarsak,

$$
-\frac{p}{\rho^2}
\left[
  \frac{\partial \rho}{\partial t} + u \frac{\partial \rho}{\partial x}
\right] +
\frac{1}{\rho} \frac{\partial p}{\partial t} -
\frac{u}{\rho} \frac{\partial p}{\partial x} +
(\gamma - 1)\frac{p}{\rho} \frac{\partial u}{\partial x} = 0
$$

Köşeli parantez içi (3)'teki formülün değişik bir formu, o zaman

$$
-\frac{p}{\rho^2} \rho \frac{\partial u}{\partial x} +
\frac{1}{\rho} \frac{\partial p}{\partial t} -
\frac{u}{\rho} \frac{\partial p}{\partial x} +
(\gamma - 1)\frac{p}{\rho} \frac{\partial u}{\partial x} = 0
$$

İlk terimdeki $\rho^2$ gider,

$$
\frac{p}{\rho} \frac{\partial u}{\partial x} +
\frac{1}{\rho} \frac{\partial p}{\partial t} -
\frac{u}{\rho} \frac{\partial p}{\partial x} +
(\gamma - 1)\frac{p}{\rho} \frac{\partial u}{\partial x} = 0
$$

Her şeyi $\rho$ ile çarparsak,

$$
p \frac{\partial u}{\partial x} +
\frac{\partial p}{\partial t} +
u \frac{\partial p}{\partial x} +
(\gamma - 1)p \frac{\partial u}{\partial x} = 0
$$

Bir basitleştirme daha,

$$
\frac{\partial p}{\partial t} +
p \frac{\partial u}{\partial x} +
\gamma p \frac{\partial u}{\partial x} = 0
$$

Böylece Euler denklemlerinin temel değişkenleri baz alan formu için gerekli üç
denklemi erişmiş oluyoruz. Tekrar listemek gerekirse, [6, sf. 299] notasyonuyla,

$$
\rho_t + u \rho_x + \rho u_x = 0
$$

$$
u_t + uu_x + (1/\rho) p_x = 0
$$

$$
p_t + \gamma p u_x + u p_x = 0
$$

Matris formuna yarı-hiperbolik sistem şu şekilde gösterilebilir,

$$
\left[\begin{array}{c}
\rho \\ u \\ p
\end{array}\right]_t +
\left[\begin{array}{ccc}
u & \rho & 0 \\
0 & u & 1/\rho \\
0 & \gamma p & u
\end{array}\right]
\left[\begin{array}{c}
\rho \\ u \\ p
\end{array}\right]_x
= 0
$$

Kaynaklar

[1] Versteeg, {\em An Introduction to CFD}

[2] Katz, {\em Introduction to Fluid Mechanics}

[3] Bayramlı, {\em Fizik, İdeal Gazlar Kanunu}

[4] Leveque, {\em Numerical Methods for Conservation Laws}

[5] Bayramlı, {\em Fizik, Gazlar, Sıvılar 1}

[6] Leveque, {\em Finite Volume Methods}

[7] Zingale, {\em Tutorial on Computational Astrophysics},
    \url{https://zingale.github.io/comp_astro_tutorial/advection_euler/euler/euler.html}

\end{document}




