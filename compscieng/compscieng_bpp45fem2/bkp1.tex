$N_i$ için biraz önceki tanımları kullanırız. $u_1,u_2$ sabittir demiştik,
türevden dışarı çıkartılabilir, yani

$$
u = u_1 N_1 + u_2 N_2
$$

turevi

$$
\frac{\ud u}{\ud X_1} = u_1 \left( \frac{\ud N_1}{\ud X_1}  \right) +
u_2 \left( \frac{\ud N_2}{\ud X_1}  \right)
$$

Matris / vektör formunda gösterebiliriz,

$$
\frac{\ud u}{\ud X_1} = \left[\begin{array}{cc}
\dfrac{\ud N_1}{\ud X_1} & \dfrac{\ud N_2}{\ud X_1} 
\end{array}\right]
\left[\begin{array}{c}
u_1 \\ u_2
\end{array}\right]
$$

(1) formülüne koyarsak,

$$
\underbrace{
  \int_{0}^{L} EA
  \left[\begin{array}{cc}
  \dfrac{\ud N_1}{\ud X_1} & \dfrac{\ud N_2}{\ud X_1} 
  \end{array}\right]
  \left( \frac{\ud N_i}{\ud X_1}  \right) \ud X_1
}_{\textrm{Direngenlik Matrisi}}
\left[\begin{array}{c} u_1 \\ u_2 \end{array}\right] =
\underbrace{
  \int_{0}^{L} p N_i \ud X_1
}_{\textrm{Düğümsel Kuvvetler}}
$$

Eşitliğin solunda gösterilen hesap bize bir matris verecek, materyel mekanığine
ona direngenlik matrisi denir. Sağ tarafta ise düğümsel kuvvetlerin vektörü var,
eğer elde dağıtık bir yük var ise ve düğümsel kuvvetleri hesaplamak istiyorsam
eşitliğin sağındaki o formülü kullanırım. Yani matris çarpı bilinmeyen $u_1,u_2$
vektör çarpımı, eşittir sağda bir vektör.. Bir lineer sistem var elimizde!
Direngenlik çarpı yer değişimi eşittir kuvvet. Bulmak istediğimiz yer
değişimleri değil mi? Üstteki sistemi çözerek onu elde edebiliriz.
